\KOMAoptions{
  numbers=noenddot,
  headings=standardclasses,
  chapterprefix=false,
  paper=a5,                   % Papierformat ist DIN B5
  fontsize=11pt
}

%% Encoding
\usepackage[T1]{fontenc}
\usepackage[utf8]{inputenc}

%% Hierfür wird das CTAN-Paket MinionPro, zusammen mit den
%% Type-1-Dateien der Minion Pro, benötigt.
\usepackage[fullfamily,textosf,mathlf,opticals,footnotefigures]{MinionPro}

%% Serifenlose
\usepackage{classico}

%% Eine Dicktengleiche
\usepackage[scaled=.8]{beramono}

%% Sprache, Anführungszeichen, Datum
\usepackage{babel}
\usepackage{csquotes}
\usepackage[cleanlook]{isodate}

%% Mikrotypographie
\usepackage{ellipsis}
\usepackage[babel,letterspace=170]{microtype}

%% Fancy tables
\usepackage{booktabs}

%% Farben
\usepackage[dvipsnames]{xcolor}

\usepackage[size=tiny]{todonotes}

%% Wird für die Kapitelziffern benötigt.
\usepackage{marginnote}

%% Sinnvoll konfigurierbare Listen
\usepackage{enumitem}

%% Text nicht vertikal strecken
\raggedbottom

%% Hurenkinder verbieten
\makeatletter
  \widowpenalty=\@M
\makeatother

%% Wir wollen, dass auch, wenn wir für Typographie alles groß oder
%% klein setzen, trotzdem die korrekten Buchstaben kopiert werden und
%% in PDF-Metadaten landen.
\usepackage{accsupp}
\makeatletter
\newcommand*{\c@pyCorrectly}[2]{\texorpdfstring{%
    \protect\BeginAccSupp{%
      method=pdfstringdef,%
      ActualText=#1%
    }%
    #2%
    \protect\EndAccSupp{}}%
  {#1}%
}

%% Akronyme
\newcommand*{\acr}[1]{\c@pyCorrectly{#1}{\textsc{\textls[40]{\MakeLowercase{#1}}}}}

%% Verschiedene Sperrungen von small caps
\newcommand*{\name}[1]{\c@pyCorrectly{#1}{\textsc{\textls[40]{\MakeLowercase{#1}}}}}
\newcommand*{\tableHead}[1]{\c@pyCorrectly{#1}{\textsc{\textls[60]{\MakeLowercase{#1}}}}}

%% Etwas mehr Spacing für spaced small caps
\SetTracking[spacing={100*,166,}]{encoding=*,shape=ssc}{15}
\newcommand*{\SAC}[1]{\c@pyCorrectly{#1}{\textls{\MakeUppercase{#1}}}}    % spaced all caps
\newcommand*{\SLSC}[1]{\c@pyCorrectly{#1}{\textssc{\MakeLowercase{#1}}}}  % spaced lowercase small caps
\makeatother

%% Paragraph
%\setkomafont{paragraph}{\sscshape\mdseries}

%% Sections
\addtokomafont{section}{\normalsize\sscshape\lsstyle\mdseries\MakeLowercase}
\renewcommand*{\sectionformat}   {\upshape\mdseries\sscshape\thesection\enskip\hspace*{2px}}
\RedeclareSectionCommand[afterskip=\baselineskip]{section}

%% Subsections
\addtokomafont{subsection}{\normalsize\mdseries\itshape}
\renewcommand*{\subsectionformat}{\upshape\mdseries\scshape\thesubsection\enskip}
\RedeclareSectionCommand[afterskip=.75\baselineskip]{subsection}

%% Chapter
\RedeclareSectionCommand[beforeskip=0\baselineskip,afterskip=1\baselineskip,afterindent=false]{chapter}
\addtokomafont{chapter}{\mdseries\normalsize\lsstyle}
\renewcommand*{\chapterformat}{\thechapter}%\normalsize\textssc{kapitel \thechapter}}
\renewcommand{\chapterlinesformat}[3]{%
  \MakeUppercase{#3}\par%
  \marginnote{\smash{\fontsize{70}{60}\selectfont #2}}
  \rule[-.15\baselineskip]{\linewidth}{.5pt}\par\nobreak
}%

%% Satzspiegel
%\KOMAoptions{DIV=10}
\areaset{288pt}{480pt}%
\setlength{\marginparwidth}{8em}%
\setlength{\marginparsep}{1.5em}%

%% Keine Kopfzeilen
\pagestyle{plain}
 
%% Footnote-Design
\usepackage{footmisc}
\setlength{\footnotemargin}{0em}
\deffootnote{0em}{0em}{%
  \textsuperscript{\normalfont\thefootnotemark\ }}


%% Einzelne Zeichen
%\fboxsep0pt
%\newcommand{\Char}[1]{\colorbox{black!10}{\strut\textcolor{Maroon}{\hspace*{1px}#1\hspace*{1px}}}}
\newcommand*{\Char}[1]{\textcolor{Maroon}{#1}}
\newcommand*{\tab}[1]{\figureversion{tab}{#1}}

\usepackage[margin    =20pt,
            font      ={footnotesize},
            format    =plain,
            labelsep  =period]{caption}

%%% Local Variables:
%%% mode: latex
%%% TeX-master: "main"
%%% End:
