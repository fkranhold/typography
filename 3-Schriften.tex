\chapter{Schriften}
\label{ch:Schriften}

Das grundlegende Arbeitsmittel der Typographie ist natürlich die
\emph{Schrift}.  Mit ihr wollen wir uns in diesem Kapitel umfassend
beschäftigen.  Einige Aspekte haben wir schon gesehen; hier gehen wir
aber weiter ins Detail.  Da sich die Gestaltung von Schrift »im
Kleinen« abspielt (im Gegensatz zur Gestaltung von bspw. ganzen
Seiten), wird das entsprechende Teilgebiet der Typographie auch als
\emph{Mikrotypographie} bezeichnet.

\section{Begriffliches und Klassifikation}

Eine \emph{Glyphe} ist die konkrete graphische Realisierung eines
(abstrakten) Schriftzeichens.  So sind bspw. \Char{a},
\Char{\textlarger{a}}, \Char{\textsmaller{a}}, \Char{\textit{a}},
\Char{\textbf{a}}, \Char{\textsf{a}}, \Char{\textsf{\textit{a}}} alles
unterschiedliche Glyphen, die den \emph{lateinischen Kleinbuchstaben
  a} darstellen.

Zusammenhängende Texte werden in einer \emph{Schriftart} gesetzt; die
darin vorhandenen Glyphen sind alle ähnlich und zueinander passend
gestaltet~-- sonst ergibt sich {U\fontfamily{ptm}\selectfont
  n\fontfamily{pplj}\selectfont r\fontfamily{lmr}\selectfont
  u\fontfamily{uop}\selectfont h\fontfamily{pcr}\selectfont e}.  Liegt
eine Schriftart in verschiedenen Varianten vor, was üblicherweise der
Fall ist, so werden diese \emph{Schriftschnitte} genannt
(s. \cref{sec:Schnitte}).  Eine solche Schriftart mit mehreren
Schnitten nennt man dann auch eine \emph{Schriftfamilie}.\footnote{Das
  ist das Beste, was ich (Philip) nach längerer Recherche zur
  Unterscheidung der Begriffe \emph{Schriftart} und \emph{-familie}
  herausbekommen zu haben glaube.  Allerdings fällt mir spontan
  (während ich diesen Text schreibe) kein Beispiel für eine Schriftart
  ein, von der es nur einen Schnitt gibt, die also keine Familie ist.}

\begin{itemize}
\item Schriftarten nach vielen Dingen klassifizierbar.  Alle
  Schriften, die uns hier interessieren, sind \emph{Antiqua}-Schriften
  (im Gegensatz zu gebrochenen Schriften, d.\,h. Fraktur und so).
\item Gibt diverse Klassen (s. bspw. \acr{DIN}~16518).  Für uns hier
  im Skript relevant: Mit Serifen und ohne (= Grotesk = Sans Serif =
  Serifenlose Linear-Antiqua).
\end{itemize}

\section{Schriftschnitte}
\label{sec:Schnitte}

\section{Liniensystem}
\label{sec:Linien}

%% Inkl. Maßeinheiten und so, und Zeilenabstand

\section{Kerning und Spacing}
\label{sec:KernSpace}

%% Hier microtype erwähnen?

\section{Ligaturen}

%% Ligaturbrecher erwähnen!

\section{Ziffern}
\label{sec:Ziffern}

%% Minuskelziffern, Majuskelziffern, Tabellenziffern / lining figures

\section{Beispiele}

{\fontfamily{pplj}\selectfont Palatino 1234 \emph{kursiv} \textsc{kapitälchen}}
%% Garamond, Palatino, Minion, Optima/Classico, Times, Helvetica

%% Jeweils auch die lustigen Namen nach DIN 16518 erwähnen!

%%% Local Variables:
%%% mode: latex
%%% TeX-master: "main"
%%% End:
