\chapter{Schriften}
\label{ch:Schriften}

Das grundlegende Arbeitsmittel der Typographie ist natürlich die
\emph{Schrift}.  Mit ihr wollen wir uns in diesem Kapitel umfassend
beschäftigen.  Einige Aspekte haben wir schon gesehen; hier gehen wir
aber weiter ins Detail.  Da sich die Gestaltung von Schrift »im
Kleinen« abspielt (im Gegensatz zur Gestaltung von bspw. ganzen
Seiten), wird das entsprechende Teilgebiet der Typographie auch als
\emph{Mikrotypographie} bezeichnet.

\section{Begriffliches und Klassifikation}

Eine \emph{Glyphe} ist die konkrete graphische Realisierung eines
(abstrakten) Schriftzeichens.  So sind bspw. \Char{a},
\Char{\textlarger{a}}, \Char{\textsmaller{a}}, \Char{\textit{a}},
\Char{\textbf{a}}, \Char{\textsf{a}}, \Char{\textsf{\textit{a}}} alles
unterschiedliche Glyphen, die den \emph{lateinischen Kleinbuchstaben
  a} darstellen.

Zusammenhängende Texte werden in einer \emph{Schriftart} gesetzt; die
darin vorhandenen Glyphen sind alle ähnlich und zueinander passend
gestaltet~-- sonst ergibt sich {U\fontfamily{ptm}\selectfont
  n\fontfamily{pplj}\selectfont r\fontfamily{lmr}\selectfont
  u\fontfamily{uop}\selectfont h\fontfamily{pcr}\selectfont e}.  Liegt
eine Schriftart in verschiedenen Varianten vor, was üblicherweise der
Fall ist, so werden diese \emph{Schriftschnitte} genannt
(s. \cref{sec:Schnitte}).  Eine solche Schriftart mit mehreren
Schnitten nennt man dann auch eine \emph{Schriftfamilie}.\footnote{Das
  ist das Beste, was ich (Philip) nach längerer Recherche zur
  Unterscheidung der Begriffe \emph{Schriftart} und \emph{-familie}
  herausbekommen zu haben glaube.  Allerdings fällt mir spontan
  (während ich diesen Text schreibe) kein Beispiel für eine Schriftart
  ein, von der es nur einen Schnitt gibt, die also keine Familie ist.}

Schriftarten werden nach Aspekten ihrer Gestaltung klassifiziert.  Wir
beschäftigen uns hier nur mit \emph{Satzschriften} (also keine
Handschriften) für das lateinische Alphabet.  Diese werden
grundsätzlich unterschieden in zwei \emph{Schriftgattungen}, nämlich
\emph{Antiqua-Schriften}\footnote{Auch für das griechische, armenische
  und kyrillische Alphabet sind Antiqua-Schriften heutzutage üblich.}
einerseits und \emph{gebrochene Schriften} andererseits: in
gebrochenen Schriften haben die Bögen der Buchstaben »Knicke«; in
Antiqua-Schriften sind sie rund.  Heutzutage werden fast alle
Druckerzeugnisse im lateinischen Alphabet in Antiqua-Schriften
gesetzt, gebrochene Schriften (bspw. \emph{Textura},
\emph{Schwabacher}, \emph{Fraktur}) sind veraltet.

\todo[inline]{Ein, zwei Sätze Historisches: Herkunft der Groß- und
  Kleinbuchstaben.}

Innerhalb der Gattung der Antiqua-Schriften werden Schriftarten weiter
in kleineren »Taxa« (um mal ein Wort aus der Biologie
zweckzuentfremden) klassifiziert:\footnote{In der Zeit des Bleisatzes
  war in Deutschland die Klassifikation gemäß \acr{DIN}~16518 üblich
  (siehe \url{https://de.wikipedia.org/wiki/DIN_16518}); diese ist
  aber einerseits kunsthistorisch fragwürdig und andererseits
  veraltet.  Wir orientieren uns im Folgenden grob an der Matrix
  Beinert (siehe
  \url{https://www.typolexikon.de/schriftklassifikation-matrix-beinert/}).}
Die Gattung wird unterteilt in \emph{Hauptschriftgruppen} und diese
jeweils in \emph{Schriftuntergruppen}.  Die Hauptschriftgruppen
unterscheiden sich dabei durch die Art der an den Glyphen vorhandenen
\emph{Serifen}, den Linien, die die Hauptstriche »abschließen«.  Die
\todo{Abbildung (vergrößert) mit Serifen?}
Klassifikation ist wie folgt:
\begin{itemize}
\item Eine Schrift mit »normalen«, nicht betonten Serifen heißt
  \emph{Antiqua} bzw. englisch \emph{\foreignlanguage{british}{serif}}
  oder \emph{\foreignlanguage{british}{roman}} (ja, der Name der
  Hauptschriftgruppe ist derselbe wie der der Gattung; diese
  Schriftarten waren nämlich die ursprünglichen Antiqua-Schriften).
  Nach ihrem historischen Entstehungszeitpunkt werden die folgenden
  drei Schriftuntergruppen unterschieden:
  \begin{itemize}
  \item \emph{Renaissance-Antiqua}, mit den Schriftnebengruppen
    \emph{Venezianische} und \emph{Französische Renaissance-Antiqua}:
    bspw. {\fontfamily{EBGaramond-OsF}\selectfont Garamond},
    {\fontfamily{pplj}\selectfont \textsmaller[.5]{Palatino}}, % URW Palladio L 
    Minion
  \item \emph{Vorklassizistische Antiqua}\footnote{Nach
      \acr{DIN}~16518: »Barock-Antiqua« (irreführend)}:
    bspw. {\fontfamily{Baskervaldx-OsF}\selectfont Baskerville},
    {\fontfamily{ptm}\selectfont Times} % Nimbus Roman No. 9 L
  \item \emph{Klassizistische Antiqua}:
    bspw. {\fontfamily{bodoni}\selectfont Bodoni},
    {\fontfamily{TheanoDidot-TOsF}\selectfont Didot},
    {\fontfamily{lmr}\selectfont Computer Modern / Latin Modern}
  \end{itemize}
\item Eine Schrift mit betonten Serifen heißt \emph{Egyptienne} bzw.
  englisch \emph{\foreignlanguage{british}{slab serif}}.\footnote{Nach
    \acr{DIN}~16518: »Serifenbetonte Linear-Antiqua«}
  (Schriftuntergruppen sind uns hier egal.)  Bspw.
  {\fontfamily{bch}\selectfont Bitstream Charter},
  {\fontfamily{ccr}\selectfont Concrete Roman},
  {\fontfamily{ucr}\selectfont Courier} % Nimbus Mono L
\item Eine Schrift ohne Serifen heißt \emph{Grotesk} bzw. englisch (ja
  ja, mit Französisch) \emph{\foreignlanguage{british}{sans
      serif}}.\footnote{Nach \acr{DIN}~16518: »Serifenlose
    Linear-Antiqua«} (Auch hier sind uns Schriftuntergruppen egal.)
  {\fontfamily{phv}\selectfont \textsmaller[.5]{Helvetica}}, % Nimbus Sans L
  {\fontfamily{pag}\selectfont \textsmaller[.5]{Avant Garde}}, % Gothic L
  Bspw.  {\fontfamily{ua1}\selectfont \textsmaller[.5]{Arial}} % URW A030
\item Die letzte Hauptschriftgruppe der Antiqua-Schriften sind
  \emph{Zierschriften}.  Das sind generell alle Antiqua-Schriften, die
  keine Antiqua, Egyptienne oder Grotesk sind. Zu ihnen gehören
  u.\,a. die sog. \emph{Antiqua-Varianten} nach \acr{DIN}~16518.  Ein
  Beispiel für letztere sind sog. \emph{Inzisen}, an Inschriften
  orientierte Schriften mit sehr dezenten Serifen, wie bspw.
  \textsf{\textsmaller[.5]{Optima / Classico}}.\footnote{Die Optima
    wird gerne --~so auch in diesem Skript~-- als Grotesk zu Palatino
    oder Minion verwendet, obwohl sie eigentlich™ keine ist.}
\end{itemize}
Die Gestaltung und Klassifikation von Schriften gleicht einem Fass
ohne Boden; wir gehen hier nicht weiter ins Detail.\footnote{Wer
  möchte, kann dazu sehr viel Weitergehendes auf Wikipedia oder im
  Typo\-lexikon (\url{https://www.typolexikon.de/}) lesen.}

\todo[inline]{Angeblich™ lesen sich Schriften mit Serifen im Fließtext
  leichter; auf jeden Fall sind sie traditioneller.}

\section{Schriftschnitte}
\label{sec:Schnitte}

%% Stärke, Lage, Breite.  Bei Lage (also gerade vs. kursiv) auch
%% Historisches™ erwähnen.

\section{Liniensystem}
\label{sec:Linien}

%% Inkl. Maßeinheiten und so, und Zeilenabstand

%% Zu Schriftgröße: 1em = Schriftgröße.  Was das mit der tatsächlichen
%% Glyphengröße zu tun hat: heutzutage sehr willkürlich; oft so was
%% wie Versalhöhe ≈ .7em (oder so?!).

%% Zeilenabstand: eigentlich sollte™ man einfach den Abstand zwischen
%% den Grundlinien messen.  Es gibt sinnvolle Standardeinstellungen,
%% bspw. in LaTeX 10pt → 12pt, 11pt → 13.6pt, 12pt → 14.5pt.  MS Word
%% und LibreOffice benutzen auch ca. 1,2-faches der Schriftgröße als
%% Grundeinstellung, das heißt da aber »einzeilig«.

\section{Kerning und Spacing}
\label{sec:KernSpace}

%% Hier microtype erwähnen?

\section{Ligaturen}

%% Ligatur = eine Glyphe für mehrere Buchstaben.  Sinnvoll, um
%% ruhigeres Schriftbild zu erreichen.  Traditionelle Ligaturen aus
%% lateinischen Texten; je nach Sprache zusätzliche.  Fun fact: ß =
%% ſ-z- oder ſ-s-Ligatur.  (Problem: die Minion hat kein ſ :( )

%% Ligaturbrecher erwähnen!

\section{Ziffern}
\label{sec:Ziffern}

%% Minuskelziffern, Majuskelziffern, Tabellenziffern / lining figures

\section{Beispiele}

{\fontfamily{pplj}\selectfont Palatino 1234 \emph{kursiv} \textsc{kapitälchen}}
%% Garamond, Palatino, Minion, Optima/Classico, Times, Helvetica

%% Jeweils auch die lustigen Namen nach DIN 16518 erwähnen!

%%% Local Variables:
%%% mode: latex
%%% TeX-master: "main"
%%% End:
