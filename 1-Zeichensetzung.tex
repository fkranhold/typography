\chapter{Zeichensetzung}

\lipsum[1]\todo{Worum geht es hier?}

\section{Anführungszeichen und Apostrophe}

Die Standardanführungszeichen sowohl im Deutschen wie im Englischen sehen immer
aus wie (gedrehte und verschobene) Kommata, also \Char{„}\,\Char{“}\,\Char{”}
sowie die einfachen Varianten \Char{‚}\,\Char{‘}\,\Char{’}. Je nachdem, wie sie
gedreht sind, werden sie als »6en« oder »9en« bezeichnet.

Im Deutschen wird die Anführungszeichenkombination »99–66« benutzt, wobei die
öffenen Anführungszeichen unten, die schließenden Jedoch oben stehen, zum
Beispiel: „Hallo!“ Bei verschachtelten Anführungszeichen werden innen einfache
Anführungszeichen, also »9–6«, benutzt: „Er sagte: ‚Hallo!{‘} und ging fort.“

Im Englischen wird (wenigstens im Britischen) die Kombination
»6–9«, beide oben, verwendet, zum Beispiel: ‘Hi!’ Bei verschachtelten
Anführungszeichen werden innen doppelte Anführungszeichen, also »66–99«,
benutzt: ‘“Hi!”, he said, and went away.’

Alternativ können im Deutschen auch die sogenannten \emph{Guillemets}
\Char{»}\,\Char{«} verwendet werden, so wie in diesem Skript. Innen werden dann
die einfachen Guillements \Char{›}\,\Char{‹} verwendet.

Wir weisen darauf hin, dass die auf einer Tastatur leichter zu erreichenden
\emph{Ersatzzeichen} \Char{"} und \Char{\textquotesingle} historische Relike aus
der Schreibmaschinenzeit und semantisch falsch sind.

Der \emph{Apostroph} \Char{’} ist von einem einfachen schließenden englischen
Anführungszeichen nicht zu unterscheiden. Man sollte die Sprache, in der man
schreibt, gut genug beherrschen, um zu wissen, wo sie hinkommen und wo
nicht.\todo{Hier Han-Solo-Meme.} Wichtig: Die freistehenden Akzentzeichen
\Char{\textasciigrave} und \Char{´} sind \emph{keine} Apostrophe!


\section{Striche}

\section{Leerzeichen}

%%% Local Variables:
%%% mode: latex
%%% TeX-master: "main"
%%% End:
