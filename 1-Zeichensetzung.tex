\chapter{Zeichensetzung}

\lipsum[1]\todo{Worum geht es hier?}

\section{Anführungszeichen und Apostrophe}

Die Standardversionen der \emph{Anführungszeichen} sowohl im Deutschen
als auch im Englischen sehen immer aus wie (gedrehte und verschobene)
Kommata, also \Char{„}\,\Char{“}\,\Char{”} sowie die einfachen
Varianten \Char{‚}\,\Char{‘}\,\Char{’}. Je nachdem, wie sie gedreht
sind, werden die einzelnen „Striche“ als »6en« oder »9en« bezeichnet
-- denn sie sehen ein wenig aus wie verkleinerte und ausgefüllte
Versionen dieser Ziffern (ein Komma \Char{,} sieht wie eine 9 aus).
\todo{Ich habe jetzt die „Konvention“ benutzt, dass \Char{»«} für
  Zitate verwendet werden und \Char{„“} für quasi „umgangssprachlich“
  verwendete Sachen.  Ist das zu viel Overkill?  -- Philip}

Im Deutschen wird die Anführungszeichenkombination »99–66« benutzt, wobei die
öffenen Anführungszeichen unten, die schließenden jedoch oben stehen, zum
Beispiel: „Hallo!“ Bei verschachtelten Anführungszeichen werden innen einfache
Anführungszeichen, also »9–6«, benutzt: „Er sagte: ‚Hallo!{‘}, und ging fort.“

Im Englischen wird (wenigstens im Britischen) die Kombination
»6–9«, beide oben, verwendet, zum Beispiel: ‘Hi!’ Bei verschachtelten
Anführungszeichen werden innen doppelte Anführungszeichen, also »66–99«,
benutzt: ‘“Hi!”, he said, and went away.’

Alternativ können im Deutschen auch die sogenannten \emph{Guillemets}
\Char{»}\,\Char{«} verwendet werden, so wie in diesem Skript. Innen werden dann
die einfachen Guillements \Char{›}\,\Char{‹} verwendet.

Wir weisen darauf hin, dass die auf einer Tastatur leichter zu erreichenden
\emph{Ersatzzeichen} \Char{"} und \Char{\textquotesingle} historische Relike aus
der Schreibmaschinenzeit und semantisch falsch sind.

Der \emph{Apostroph} \Char{’} ist von einem einfachen schließenden englischen
Anführungszeichen nicht zu unterscheiden. Man sollte die Sprache, in der man
schreibt, gut genug beherrschen, um zu wissen, wo sie hinkommen und wo
nicht.\todo{Hier Han-Solo-Meme.} Wichtig: Die freistehenden Akzentzeichen
\Char{\textasciigrave} und \Char{´} sind \emph{keine} Apostrophe!


\section{Striche}

Es gibt verschiedene Arten von (horizontalen) \emph{Strichen}, die in
Texten aller Art auftauchen.  Hier eine kleine Auswahl:
\Char{-\,--\,---\,$-$\,$=$}.  Diese Striche haben verschiedene
Aufgaben, die gerne verwechselt werden.  Typographisch unterscheiden
wir die Striche üblicherweise nach ihrer Länge.

\begin{description}
\item[Viertelgeviertstrich]
\end{description}

\section{Leerzeichen}

%%% Local Variables:
%%% mode: latex
%%% TeX-master: "main"
%%% End:
