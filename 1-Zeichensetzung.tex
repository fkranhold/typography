\chapter{Zeichensetzung}

\lipsum[1]\todo{Worum geht es hier?}

\section{Anführungszeichen und Apostrophe}

Die Standardversionen der \emph{Anführungszeichen} sowohl im Deutschen
als auch im Englischen sehen immer aus wie (gedrehte und verschobene)
Kommata, also \Char{„}\,\Char{“}\,\Char{”} sowie die einfachen
Varianten \Char{‚}\,\Char{‘}\,\Char{’}.  Je nachdem, wie sie gedreht
sind, werden die einzelnen »Striche« als »6en« oder »9en« bezeichnet
-- denn sie sehen ein wenig aus wie verkleinerte und ausgefüllte
Versionen dieser Ziffern (ein Komma \Char{,} sieht wie eine 9 aus).

Im Deutschen wird die Anführungszeichenkombination »99–66« benutzt, wobei die
öffenen Anführungszeichen unten, die schließenden jedoch oben stehen, zum
Beispiel: „Hallo!“ Bei verschachtelten Anführungszeichen werden innen einfache
Anführungszeichen, also »9–6«, benutzt: „Er sagte: ‚Hallo!{‘}, und ging fort.“

Im Englischen wird (wenigstens im Britischen) die Kombination »6–9«,
beide oben, verwendet, zum Beispiel: \foreignlanguage{british}{‘Hi!’}
Bei verschachtelten Anführungszeichen werden innen doppelte
Anführungszeichen, also »66–99«, benutzt:
\foreignlanguage{british}{‘“Hi!”, he said, and went away.’}

Alternativ können im Deutschen auch die sogenannten \emph{Guillemets}
\Char{»}\,\Char{«} verwendet werden, so wie in diesem Skript.  Innen
werden dann die einfachen Guillements \Char{›}\,\Char{‹} verwendet.

Wir weisen darauf hin, dass die auf einer Tastatur leichter zu
erreichenden \emph{Ersatzzeichen} \Char{"} und \Char{\textquotesingle}
(bei deutscher \acr{QWERTZ}-Belegung \keys{\shift+2}
bzw. \keys{\shift+\#}) historische Relikte aus der
Schreibmaschinenzeit und semantisch falsch sind.
\todo{Auch hier schon Akzente erwähnen?  (Leute machen die wildesten
  Dinge …)  --~Philip}

Der \emph{Apostroph} \Char{’} ist von einem einfachen schließenden englischen
Anführungszeichen nicht zu unterscheiden.  Man sollte die Sprache, in der
man schreibt, gut genug beherrschen, um zu wissen, wo Apostrophe hinkommen und
wo nicht.\todo{Hier Han-Solo-Meme.}  Wichtig: Die freistehenden Akzentzeichen
\Char{\textasciigrave} und \Char{´} sind, genau wie das Ersatzzeichen
\Char{\textquotesingle}, \emph{keine} Apostrophe!


\section{Striche}

Es gibt verschiedene Arten von (horizontalen) \emph{Strichen}, die in
Texten aller Art auftauchen.  Hier eine kleine Auswahl:
\Char{-\,--\,---\,$-$\,$=$}.  Diese Striche haben verschiedene
Aufgaben, die gerne verwechselt werden.  Typographisch unterscheiden
wir die Striche üblicherweise nach ihrer Länge.

\paragraph{Viertelgeviertstrich}
Der Viertelgeviertstrich \Char{-} ist der kürzeste der üblichen
Striche.  Er wird auf Deutsch sowie auf Englisch als
\emph{Trennstrich} bei Worttrennungen an Zeilenumbrüchen verwendet.
Auf Deutsch ist er außerdem der \emph{Bindestrich} in nicht
zusammenge\-schriebenen Komposita, bspw. in »Anti-Terror-Anschlag«.
Dieses grammatikalische Phänomen nennt man \emph{Durchkopplung}; es
betrifft auch die Zusammensetzung von Eigennamen aus mehreren Wörtern
mit weiteren Wörtern zu einem Kompositum: bspw. wird aus »Johann
Sebastian Bach« und »Straße« die »Johann-Sebastian-Bach-Straße«.

Im Englischen gibt es üblicherweise \emph{keine} Durchkopplung, außer
bei zusammengesetzten Adjektiven: es heißt »apple tree«, »anti-terror
attack« oder »Johann Sebastian Bach Street«.

Im Deutschen werden auch Dinge, die nach mehreren Personen benannt
sind, als normales Kompositum betrachtet und dementsprechend mit
Bindestrich gesetzt.  Dieses Skript könnte man bspw. das
»Kranhold-Schwartz-Skript« nennen.  (Auf Englisch ist das anders,
siehe unten.)

\paragraph{Halbgeviertstrich}
Der \emph{Halbgeviertstrich} \Char{--} ist doppelt so lang wie der
Viertelgeviertstrich.  Er ist der \emph{Bis-Strich} in Zeitraumangaben
wie »in den Jahren 2012--2023« oder »geöffnet Mo.--Fr. 9--18 Uhr«.
Dabei ist zu beachten, dass er \emph{nicht} in der Phrase »von … bis
…« das »bis« ersetzen darf: »von 11--13 Uhr« ist falsch.  Auch in
Streckenangaben steht ein Halbgeviertstrich: »Auf der Strecke
Köln--Karlsruhe fährt der \acr{ICE} 203.«

Er kann auch als Aufzählungszeichen verwendet werden
(»Spiegelstrich«):
\begin{itemize}[label=--,nosep]
\item Es gibt sone und
\item solche, und
\item dann gibt's noch ganz andere!
\end{itemize}
Meistens sieht das aber nicht so gut aus und Aufzählungs\emph{punkte}
sind besser.

Im Deutschen ist der Halbgeviertstrich, umschlossen von Leerzeichen,
auch der \emph{Gedankenstrich}: »Ein Gedankenstrich sieht ganz gut
aus~-- manchmal zumindest.«  Dabei ist darauf zu achten, wie sich die
Striche an Zeilenumbrüchen verhalten: Wenn ein Gedankenstrich einen
Satz in zwei Teile teilt, soll höchstens \emph{nach} ihm umgebrochen
werden (er »klebt« also am ersten Satzteil); wenn Gedankenstriche
einen Einschub umschließen, soll höchstens \emph{vor} dem vorderen und
\emph{nach} dem hinteren umgebrochen werden (sie »kleben« also am
Einschub).  Das sieht dann also wie folgt aus:  Manche --~Dinge
umschließende~-- Striche kleben an Wörtern, wohingegen~--
aber Stopp, das führt jetzt zu weit.

Auf Englisch wird traditionell der Halbgeviertstrich benutzt, um nach
merheren Personen benannte Dinge zu schreiben.  Dieses Skript wären
dann also bspw. die »\foreignlanguage{british}{Kranhold--Schwartz
  course notes}«.  Manchmal (insb. in \acr{US}-amerikanischen /
nordamerikanischen Kontexten, aber nicht nur da) wird dafür
stattdessen aber wie auf Deutsch ein Viertelgeviertstrich verwendet.

\paragraph{Geviertstrich}
Der \emph{Geviertstrich} \Char{---} wird im Deutschen üblicherweise
nicht verwendet; eine Ausnahme stellen manchmal Quellenangaben bei
Zitaten dar:
\begin{displayquote}
  Musik ist wie ein Hamburger, die Noten sind die \mbox{Gurken}.

  \quad---~J.\,S.\,Bach
\end{displayquote}

In englischer Typographie wird oft ein Geviertstrich ohne
umschließende Leerzeichen als Gedankenstrich verwendet:
»\foreignlanguage{british}{This looks quite interesting---but also
  somewhat peculiar for Germans.}»  Oft wird aber auch im Englischen
ein Halbgeviertstrich mit Leerzeichen verwendet.

\paragraph{Minuszeichen}
Das im Formelsatz verwendete \emph{Minuszeichen} \Char{$-$} ist
natürlich semantisch verschieden von den vorher beschriebenen
Text-Strichen; in den meisten Schriftarten sieht es auch anders aus.
Seine Breite entspricht der des Pluszeichens, und es ist vertikal
zentriert.  Zum Beispiel ist $1 + 2 - 3 = 0$.

\paragraph{Gleichheitszeichen}
Auch das \emph{Gleichheitszeichen} \Char{$=$} besteht natürlich aus
horizontalen Strichen~-- nämlich \emph{Zwillings-Linien}, wie sein
Erfinder \textls[15]{\textsc{Robert Recorde} (ca. 1510--1558)}
schrieb:
\begin{displayquote}
  \foreignlanguage{british}{And to avoide the tediouse repetition of
    these woordes : is equalle to : I will sette as I doe often in
    woorke use, a paire of paralleles, or Gemowe lines of one lengthe,
    thus: $=$, bicause noe .2. thynges, can be moare equalle.}

  \quad---~Robert Recorde, \emph{The Whetstone of Witte} (1557)
\end{displayquote}

In gebrochenen Schriften sieht außerdem der Trennstrich einem
Gleichheitszeichen ähnlich.  Wir beschäftigen uns hier aber nicht mit
gebrochenen Schriften.

\section{Leerzeichen}

%%% Local Variables:
%%% mode: latex
%%% TeX-master: "main"
%%% End:
