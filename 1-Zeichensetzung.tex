\chapter{Zeichensetzung}

\lipsum[1]\todo{Worum geht es hier?}

\section{Anführungszeichen und Apostrophe}

Die Standardanführungszeichen sowohl im Deutschen wie im Englischen sehen immer
aus wie (gedrehte und verschobene) Kommata, also \Char{„}\,\Char{“}\,\Char{”}
sowie die einfachen Varianten \Char{‚}\,\Char{‘}\,\Char{’}. Je nachdem, wie sie
gedreht sind, werden sie als »6en« oder »9en« bezeichnet.

Im Deutschen 

Dann gibt es noch »…«

\section{Striche}

Es gibt verschiedene Arten von (horizontalen) \emph{Strichen}, die in
Texten aller Art auftauchen.  Hier eine kleine Auswahl:
\Char{-\,--\,---\,$-$\,$=$}.  Diese Striche haben verschiedene
Aufgaben, die gerne verwechselt werden.  Typographisch unterscheiden
wir die Striche üblicherweise nach ihrer Länge.

\begin{description}
\item[Viertelgeviertstrich]
\end{description}

\section{Leerzeichen}

%%% Local Variables:
%%% mode: latex
%%% TeX-master: "main"
%%% End:
