\chapter{Zeichensetzung}

Dieses erste Kapitel beschäftigt sich streng genommen noch nicht mit
Typographie, sondern lediglich mit der semantisch korrekten Verwendung von
Sonderzeichen in deutschen und englischen Texten. Für diese Problematik genügt
es also, einen Text als reine Kette von wohldefinierten Zeichen zu betrachten.
Durch den Unicode-Standard ist jedes dieser Zeichen durch einen
eindeutigen, üblicherweise hexadezimal angegebenen \emph{Codepoint} (z.\,B. \codepoint{0061} für »a«) festgelegt.

Die korrekte Verwendung grundlegender Satzzeichen (Punkt, Komma und Semikolon sowie
Frage- und Ausrufezeichen) setzen wir als bekannt voraus. Bei einigen
anderen Sonderzeichen gibt es jedoch Subtilitäten, die oft missachtet werden.

\section{Anführungszeichen}

Die Standardversionen der \emph{Anführungszeichen} sowohl im Deutschen als auch
im Englischen sehen aus wie (gedrehte und verschobene) Kommata, also
\Char{„}\,\Char{“}\,\Char{”} sowie die einfachen Varianten
\Char{‚}\,\Char{‘}\,\Char{’}.  Je nachdem, wie sie gedreht sind, werden die
einzelnen Striche als »6en« oder »9en« bezeichnet -- denn sie sehen
aus wie verkleinerte und ausgefüllte Versionen dieser Ziffern (ein Komma
\Char{,} sieht wie eine 9 aus).\looseness-1

Im Deutschen wird die Anführungszeichenkombination »99–66« benutzt, wobei die
öffenen Anführungszeichen unten, die schließenden jedoch oben stehen, zum
Beispiel: „Hallo!“ Bei verschachtelten Anführungszeichen werden innen einfache
Anführungszeichen, also »9–6«, benutzt: „Er sagte: ‚Hallo!{‘}, und ging fort.“
Wir merken an, dass das einfache öffnende deutsche Anführungszeichen \Char{‚}
(\codepoint{201a}) einen anderen Unicode-Codepoint als das Komma \Char{,}
(\codepoint{002c}) besitzt.\looseness-1

Im Englischen wird (wenigstens im Britischen) die Kombination »6–9«,
beide oben, verwendet, zum Beispiel: \foreignlanguage{british}{‘Hi!’}
Bei verschachtelten Anführungszeichen werden innen doppelte
Anführungszeichen, also »66–99«, benutzt:
\foreignlanguage{british}{‘“Hi!”, he said, and went away.’}

Alternativ können im Deutschen auch die sogenannten \emph{Guillemets}
\Char{»}\,\Char{«} verwendet werden, so wie in diesem Skript.  Innen werden dann
die einfachen Guillements \Char{›}\,\Char{‹} verwendet. Allerdings sollte man
sich für eine Variante entscheiden und diese konsequent benutzen.

Wir weisen darauf hin, dass die auf einer Tastatur leichter zu erreichenden
\emph{Ersatzzeichen} \Char{\char"22} und \Char{\textquotesingle} (bei deutscher
\acr{QWERTZ}-Belegung \keys{\shift+2} bzw. \keys{\shift+\#}) Überbleibsel
aus der Schreibmaschinenzeit und semantisch falsch sind. Die freistehenden
Akzent\-zeichen \Char{\textasciigrave} und \Char{´} sind ebenfalls keine
Anführungszeichen.

Viele Programme (wie etwa \acr{MS} Word) ersetzen eingegebene Ersatzzeichen
automatisch durch die korrekten Anführungszeichen in der eingestellten
Sprache. In anderen Situationen ist es jedoch notwendig, die Zeichen manuell
einzugeben. Auf einer Linux-Tastatur mit deutschem Layout gibt es hierfür
praktische Tastenkombinationen; unter Windows kann man bei gehaltener
\keys{Alt}-Taste nach einem \keys{{+}} den entsprechenden Unicode-Codepoint
eingeben und anschließend die \keys{Alt}-Taste lösen. In
\cref{tab:quotationMarks} finden sich die entsprechenden Tastenkombinationen und
Codepoints.

\begin{table}
  \centering
  \renewcommand{\arraystretch}{1.2}
  \begin{tabular}{clc}
    \toprule
    \tableHead{Zeichen} & \tableHead{Linux-Tastatur} & \tableHead{Codepoint}\\
    \midrule
    \Char{„} & \keys{\AltGr+v} & \codepoint{201e}\\
    \Char{“} & \keys{\AltGr+b} & \codepoint{201c}\\
    \Char{”} & \keys{\AltGr+n} & \codepoint{201d}\\
    \Char{‚} & \keys{\AltGr+\shift+v} & \codepoint{201a}\\
    \Char{‘} & \keys{\AltGr+\shift+b} & \codepoint{2018}\\
    \Char{’} & \keys{\AltGr+\shift+n} & \codepoint{2019}\\
    \Char{»} & \keys{\AltGr+y} & \codepoint{00bb}\\
    \Char{«} & \keys{\AltGr+x} & \codepoint{00ab}\\
    \Char{›} & \keys{\AltGr+\shift+y} & \codepoint{203a}\\
    \Char{‹} & \keys{\AltGr+\shift+x} & \codepoint{2039}\\
    \bottomrule
  \end{tabular}
  \caption{Anführungszeichen und wie sie erzeugt werden können}
  \label{tab:quotationMarks}
\end{table}

\section{Apostrophe}

Der \emph{Apostroph} \Char{’} sieht genauso aus wie ein einfaches schließendes
englisches Anführungszeichen. Auch wenn es sich semantisch von diesem
unterscheidet, hat es den gleichen Codepoint und wird daher genauso erzeugt.
Man sollte die Sprache, in der man schreibt, gut genug beherrschen, um zu
wissen, wo Apostrophe hinkommen und wo nicht.  Wichtig: Die freistehenden
Akzentzeichen \Char{\textasciigrave} und \Char{´} sind, genau wie das
Ersatzzeichen \Char{\textquotesingle}, \emph{keine} Apostrophe!\looseness-1

\section{Striche}
\label{sec:Striche}

Es gibt verschiedene Arten von (horizontalen) \emph{Strichen}, die in Texten
aller Art auftauchen.  Hier eine kleine Auswahl:
\Char{-}\,\Char{--}\,\Char{---}\,\Char{$-$}\,\Char{$=$}.  Diese Striche haben
verschiedene Aufgaben, die gerne verwechselt werden.  Typographisch
unterscheiden wir die Striche nach ihrer Länge.

Diese Längen werden als Anteil an einem \emph{Geviert} angegeben~--
eine Einheit, die auf die Zeit des Bleisatzes zurückgeht.  Uns soll
hier folgende Faustregel genügen: Ein Geviert ist, je nach Schriftart,
etwa das \smallfrac43-fache der Höhe eines Großbuchstabens
(vgl. \cref{sec:Linien}).

\subsection{Viertelgeviertstrich}
\label{subsec:Viertel}

Der Viertelgeviertstrich \Char{-} ist der kürzeste aller üblichen
Striche.  Er wird auf Deutsch sowie auf Englisch als \emph{Trennstrich} bei
Worttrennungen an Zeilenumbrüchen verwendet.  Auf Deutsch ist er außerdem der
\emph{Bindestrich} in nicht zusammenge\-schriebenen Komposita, bspw. in
»Anti-Terror-Anschlag«.  Dieses grammatikalische Phänomen nennt man
\emph{Durchkopplung}; es betrifft auch die Zusammensetzung von Eigennamen aus
mehreren Wörtern mit weiteren Wörtern zu einem Kompositum: bspw. wird aus
»Johann Sebastian Bach« und »Straße« die »Johann-Sebastian-Bach-Straße«.

Im Englischen gibt es üblicherweise \emph{keine} Durchkopplung, außer
bei zusammengesetzten Adjektiven: es heißt »apple tree«, »anti-terror
attack« oder »Johann Sebastian Bach Street«.

Im Deutschen werden auch Dinge, die nach mehreren Personen benannt
sind, als normales Kompositum betrachtet und dementsprechend mit
Bindestrich gesetzt.  Dieses Skript könnte man bspw. das
»Kranhold-Schwartz-Skript« nennen.  (Auf Englisch ist das anders,
siehe unten.)

\subsection{Halbgeviertstrich}
\label{subsec:halbgeviert}
Der Halbgeviertstrich \Char{--} ist doppelt so lang wie der
Viertelgeviertstrich.  Er ist der \emph{Bis-Strich} in Zeitraumangaben
wie »in den Jahren 2012--2023« oder »geöffnet Mo.--Fr. 9--18 Uhr«.
Dabei ist zu beachten, dass er \emph{nicht} in der Phrase »von … bis
…« das »bis« ersetzt: »von 11--13 Uhr« ist falsch.  Auch in
Streckenangaben steht ein Halbgeviertstrich: »Auf der Strecke
Köln--Karlsruhe fährt der \acr{ICE} 203.«

Er kann auch als Aufzählungszeichen verwendet werden (dann nennt man
ihn manchmal »Spiegelstrich«):
 \begin{itemize}[label=--]
 \item Es gibt sone und
 \item solche, und
 \item dann gibt's noch ganz andere!
 \end{itemize}
 Meistens sieht das aber nicht so gut aus und Aufzählungs\emph{punkte}
 \Char{\textbullet} sind besser.

Im Deutschen ist der Halbgeviertstrich, umschlossen von Leerzeichen,
auch der \emph{Gedankenstrich}: »Ein Gedankenstrich sieht ganz gut
aus~-- manchmal zumindest.«  Dabei ist darauf zu achten, wie sich die
Striche an Zeilenumbrüchen verhalten: Wenn ein Gedankenstrich einen
Satz in zwei Teile teilt, soll höchstens \emph{nach} ihm umgebrochen
werden (er »klebt« also am ersten Satzteil); wenn Gedankenstriche
einen Einschub umschließen, soll höchstens \emph{vor} dem vorderen und
\emph{nach} dem hinteren umgebrochen werden (sie »kleben« also am
Einschub).  Das sieht dann also wie folgt aus:  Manche --~Dinge
umschließende~-- Striche kleben an Wörtern, wohingegen~--
aber Stopp, das führt jetzt zu weit.

Auf Englisch wird traditionell der Halbgeviertstrich benutzt, um nach
mehreren Personen benannte Dinge zu schreiben.  Dieses Skript wären
dann also bspw. die »\foreignlanguage{british}{Kranhold--Schwartz
  course notes}«.  Manchmal (insb. in \acr{US}-amerikanischen /
nordamerikanischen Kontexten, aber nicht nur da) wird dafür
stattdessen aber wie auf Deutsch ein Viertelgeviertstrich verwendet.

\subsection{Geviertstrich}

Der Geviertstrich \Char{---} wird im Deutschen üblicherweise
nicht verwendet; eine Ausnahme stellen Quellenangaben bei
Zitaten dar:
\begin{displayquote}
  Musik ist wie ein Hamburger,\\
  die Noten sind die \mbox{Gurken}.

  \quad---~J.\,S.\,Bach
\end{displayquote}
%
In englischer Typographie wird oft ein Geviertstrich ohne umschließende
Leerzeichen als Gedankenstrich verwendet: »\foreignlanguage{british}{This looks
  quite interesting---but also somewhat peculiar for Germans.}« Oft wird aber
auch im Englischen ein Gedankenstrich durch einen Halbgeviertstrich mit
Leerzeichen realisiert.

Der Viertelgeviertstrich hat auf einer \acr{QWERTZ}-Tastatur eine eigene Taste.
In \cref{tab:striche} steht, wie man Halbgeviert- und Geviertstriche auf einer
Tastatur erzeugen kann.

\begin{table}
  \centering
  \renewcommand{\arraystretch}{1.2}
  \begin{tabular}{lclc}
    \toprule
    \tableHead{Zeichen} & & \tableHead{Linux-Tastatur} & \tableHead{Codepoint}\\
    \midrule
%    \midrule
    Halbgeviertstrich & \Char{–} & \keys{\AltGr+-} & \codepoint{2013}\\
    Geviertstrich & \Char{—} & \keys{\AltGr+\shift+-} & \codepoint{2014}\\
%    Minuszeichen & \Char{\textminus} & & \codepoint{2212}\\
    \bottomrule
  \end{tabular}
  \caption{Striche und wie sie erzeugt werden können}
  \label{tab:striche}
\end{table}

\subsection{Minus- und Gleichheitszeichen}

Das im Formelsatz verwendete \emph{Minuszeichen} \Char{$-$} ist
semantisch verschieden von den vorher beschriebenen Text-Strichen; in
den meisten Schriftarten sieht es auch anders aus.  Seine Breite
entspricht der des Pluszeichens, und es ist vertikal zentriert:
$1+2-3=0$.

Auch das \emph{Gleichheitszeichen} \Char{$=$} besteht natürlich aus horizontalen
Strichen~-- nämlich \emph{Zwillings-Linien}, wie sein Erfinder
\name{Robert Recorde} (ca. 1510--1558) schrieb:
\begin{displayquote}
  \foreignlanguage{british}{And to avoide the tediouse repetition of
    these woordes : is equalle to : I will sette as I doe often in
    woorke use, a paire of paralleles, or Gemowe lines of one lengthe,
    thus: $=$, bicause noe .2. thynges, can be moare equalle.}

  \quad---~Robert Recorde, \emph{The Whetstone of Witte} (1557)
\end{displayquote}

Für diese Formelzeichen gibt es bei \LaTeX\ eine weitere
Unterscheidung zwischen zwei Varianten: es gibt jeweils eine für
Verwendung im Text sowie eine für Verwendung in Formelsatz.  Das
korrekte Minus- und Gleichheitszeichen in der Schriftart dieses
Dokuments im Text sind \Char{\textminus} bzw. \Char{=}; die des
Formelsatzes sind \Char{$-$} bzw. \Char{$=$}.

% In gebrochenen Schriften sieht außerdem der Trennstrich einem
% Gleichheitszeichen ähnlich.  Wir beschäftigen uns hier aber nicht mit
% gebrochenen Schriften.
% Der Viertelgeviertstrich hat auf einer \acr{QWERTZ}-Tastatur eine eigene Taste;
% das Gleichheitszeichen kann mit \keys{\shift+0} erzeugt werden. Wie die übrigen
% Striche erzeugt werden, steht in \cref{tab:striche}. Will man explizit das
% Minuszeichen als UnicodeFür das Minuszeichen gibt es leider keine praktische
% Tastenkombination unter Linux; in einigen Desktopumgebungen (wie etwa
% \acr{GNOME}) kann nach \keys{Strg+\shift+u} der Codepoint eingegeben werden, der
% danach mit \keys{\return} bestätigt werden muss.  In \LaTeX{} taucht das
% Minuszeichen ohnehin zumeist in der Mathematikumgebung auf, in einer solchen
% wird es mit einem Viertelgeviertstrich codiert.



\section{Leerzeichen}

Das gewöhnliche Leerzeichen (\codepoint{0020}) hat üblicherweise die Breite eines
Viertelgevierts. Man sollte darauf achten, Leerzeichendopplungen zu vermeiden;
solche fallen oft erst beim genauen Gegenlesen auf. (In \LaTeX\ oder \acr{HTML}
spielen sie zum Glück keine Rolle, weil sie beim Kompilieren bzw. Rendern
ignoriert werden.)

Möchte man (wie z.\,B. bei Gedankenstrichen) verhindern, dass bei einem
ganz konkreten Leerzeichen ein Zeilenumbruch erfolgt, so verwendet man an dieser
Stelle ein \emph{(umbruch-)geschütztes Leerzeichen} gleicher Breite
(\codepoint{00a0}). In \LaTeX\ wird dies durch \verb!~! realisiert.

Das \emph{schmale Leerzeichen} (\codepoint{2009}) ist zwischen \smallfrac18\ und
\smallfrac16\ eines Gevierts breit und stets geschützt. In \LaTeX\ kann
es durch \verb!\,! erzeugt werden. Es kommt u.\,a. zum Einsatz in
\begin{itemize}
\item Abkürzungen wie etwa d.\,h., z.\,B. oder i.\,d.\,R.,
\item Datumsangaben wie etwa 28.\,12.\,2024,
\item Abkürzungen mit Nummern wie etwa §\,3, Thm.\,4.7,
\item Zahlen mit Einheitenzeichen wie etwa 87\,km oder 37\,\%,
\item großen Zahlen wie 12\,345.
\end{itemize}

Auf Englisch gibt es folgende zwei Besonderheiten:
\begin{itemize}
\item In Abkürzungen wie »\foreignlanguage{british}{i.e.}« steht
  üblicherweise \emph{kein} Leerzeichen, auch kein schmales.
\item Traditionell werden die Leerzeichen zwischen Sätzen ein wenig
  verbreitert (sog. \emph{\foreignlanguage{british}{English spacing}},
  im Gegensatz zum im Deutschen üblichen
  \emph{\foreignlanguage{british}{French spacing}}):

  »\foreignlanguage{british}{I went there.  After that, I went
    somewhere else.}«

  »\foreignlanguage{british}{\frenchspacing I went there.  After that,
    I went somewhere else.}«
\end{itemize}


\section{Auslassungspunkte}

Der \emph{Dreipunkt} (auch die \emph{Ellipse}) \Char{…}, mit dem
Auslassungspunkte gesetzt werden, ist nicht etwa die Aneinanderreihung
von drei (oder noch schlimmer: vier oder 17) Punkten \Char{{.}{.}{.}},
sondern ein eigenständiges Zeichen (\codepoint{2026}, Linux-Tastatur
\keys{\AltGr+.}).  Wird ein Satzteil ausgelassen, so wird der
Dreipunkt mit einem geschützten Leerzeichen angehängt; wird ein
Wortteil ausgelassen, bekommt er natürlich kein Leerzeichen: »Manche
Leute mögen denken, das sei ganz große Sch… ganz großer
Scheibenkleister.  Aber so sieht es einfach besser aus~…«

%%% Local Variables:
%%% mode: latex
%%% TeX-master: "main"
%%% End:
