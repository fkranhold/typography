\chapter{Hervorhebungen}

Oft ist man in der Situation, dass der eigentliche Text in einer bestimmten
Schrift gesetzt ist, aber beispielsweise einzelne Wörter oder
Gliederungselemente (wie Titel und Überschriften) zur besseren Übersichtlichkeit
\emph{hervorgehoben} werden sollen. Hierfür gibt es verschiedene Möglichkeiten, die
je nach Situation geeigneter als andere sein können.

\section{Kursivsatz}

Die wohl üblichste Methode, Passagen im Fließtext hervorzuheben, ist, sie
\emph{kursiv} zu setzen. Diese Hervorhebung ist dezent und seit vielen
Jahrhunderten etabliert. Fast alle Schriftarten bringen eigene Schriftschnitte
für kursiven Text (siehe \cref{sec:Schnitte}) mit. Sollte ein kursiver
Schriftschnitt fehlen, könnte das Textsatzprogramm versuchen, den Kursivsatz
künstlich durch \textsl{Schrägen} zu erzeugen; davon ist abzuraten.

\section{Fettdruck}

Eine heutzutage sehr verbreitete, aber noch gar nicht so alte Methode der
Hervorhebung ist die Verwendung von {\fontseries{eb}\selectfont {Fett-}} oder
besser \textbf{Mittelfettbuchstaben}. Man sollte bedenken, dass diese
Auszeichnungsart sehr hervorstricht. Üblicher ist es, Gliederungselemente wie
Überschriften mit fetten Buchstaben zu setzen. In einem sehr klassischen Layout
(wie dem dieses Skripts) taucht diese Auszeichnung kaum auf. Eine
\emph{{\fontseries{b}\selectfont{Kombination}}} von fett und kursiv ist zwar
möglich, sticht aber extrem hervor und sollte deshalb mit äußerster Vorsicht
eingesetzt werden.

\section{Kapitälchen}

Viele Schriftarten besitzen eine Variante der Großbuchstaben, die so hoch wie
übliche Kleinbuchstaben ist. Diese Glyphen heißen \textsc{kapitälchen}. Man
bemerke, dass sich diese Zeichen deutlich von einer reinen Skalierung der
Großbuchstaben unterscheiden, zum Vergleich: \textsmaller[2]{KAPITÄLCHEN}; der
wohl größte Unterschied ist die deutlich zu geringe Strichdicke. Bevor man
Kapitälchen verwendet, sollte man also prüfen, ob die eingesetzte Schriftart
solche enthält, da man sonst Gefahr läuft, Kapitälchen durch Skalierung zu
imitieren.\looseness-1

Bei Kapitälchen wird nicht zwischen Groß- und Kleinbuchstaben
unterschieden. Dennoch ist es geschieht es oft, dass Großbuchstaben in einer
Kapitälchenumgebung mit normalen Großbuchstaben gesetzt werden, also
\textsc{Kapitälchen}. Gerade hier ist es wichtig, »echte« Kapitälchen zu
nutzen; ansonsten erhält man so etwas wie K\textsmaller[2]{APITÄLCHEN}, also
unterschiedliche Strichdicken im selben Wort.

Kapitälchen werden häufig \emph{gesperrt}, d.\,h. der Abstand zwischen den
einzelnen Buchstaben wird künstlich vergrößert, zum Vergleich:
\textls[50]{\textsc{kapitälchen}}. Kleinbuchstaben will man hingegen nicht
\textls[50]{sperren}.

Kapitälchen können an verschiedenen Stellen zum Einsatz kommen, und je nach
Anwendung bietet sich ein anderer Sperrfaktor an. In diesem Skript werden
Kapitälchen unter anderem für die Überschriften der Abschnitte verwendet. Nicht
selten werden Namen wie \textls[15]{\textsc{immanuel kant} (1724–1804)} mit
\todo{Tatsächlich hier die Jahreszahlen mit-sperren?  --~Philip}
Kapitälchen hervorgehoben.\looseness-1

\section{Exkurs: Akronyme}

Es ist ratsam, \emph{Akronyme} wie etwa \acr{spd} oder \acr{PHP} mit leicht
gesperrten Kapitälchen zu setzen, damit sie nicht, wie etwa SPD und PHP,
hervorstechen. Bei Akronymen, die Zahlen enthalten (wie etwa \acr{3D},
\acr{C-3PO} oder \acr{BW\kern-.3px V\,29}), müssen dann aber
\emph{Minuskelziffern} (siehe \cref{sec:Ziffern}) verwendet, sonst erhält man
\acr{\figureversion{lf}{3}d}, \acr{C-\figureversion{lf}{3}PO} und
\acr{bw\kern-.3px v\,\figureversion{lf}{29}}.

\section{Versalien}

Die einfachste, aber deutlichste (oder besser: kompromissloseste) Art
der Hervorhebung ist die Verwendung von \textls[60]{VERSALIEN} (also
Großbuchstaben), die dann gesperrt werden sollten. Wir verwenden sie für
Kapitelüberschriften und den Skripttitel.  Diese Auszeichnungsform ist etwas aus
der Mode gekommen, findet sich aber in alten Texten sehr häufig.\looseness-1

\section{Unterstreichungen}

Unterstreichungen sind \underline{Mist}, vor allem, wenn Buchstaben vorkommen,
die \underline{Unterlänge} haben. Mogelt man und ignoriert Kollisionen, so wird
es auch nicht besser: \underline{\smash{Unterlänge}}.


%% - Kursiv
%% - Unterstreichen (mist)
%% - Fett (modern)
%% - Kapitälchen
%% - Versalien
%% - Sperren von Kapitälchen und Versalien

%%% Local Variables:
%%% mode: latex
%%% TeX-master: "main"
%%% End:
