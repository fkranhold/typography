\documentclass[british,ngerman]{scrbook}

\author{Florian Kranhold, Philip Schwartz}
\title {Typographie}

\KOMAoptions{
  numbers=noenddot,
  headings=standardclasses,
  chapterprefix=false,
  paper=a5,                   % Papierformat ist DIN B5
  fontsize=11pt
}

%% Encoding
\usepackage[T1]{fontenc}
\usepackage[utf8]{inputenc}

%% Hierfür wird das CTAN-Paket MinionPro, zusammen mit den
%% Type-1-Dateien der Minion Pro, benötigt.
\usepackage[fullfamily,textosf,mathlf,opticals,footnotefigures]{MinionPro}

%% Serifenlose
\usepackage{classico}

%% Eine Dicktengleiche
\usepackage[scaled=.8]{beramono}

%% Sprache, Anführungszeichen, Datum
\usepackage{babel}
\usepackage{csquotes}
\usepackage[cleanlook]{isodate}

%% Mikrotypographie
\usepackage{ellipsis}
\usepackage[babel,letterspace=170]{microtype}

%% Fancy tables
\usepackage{booktabs}

%% Farben
\usepackage[dvipsnames]{xcolor}

\usepackage[size=tiny]{todonotes}

%% Wird für die Kapitelziffern benötigt.
\usepackage{marginnote}

%% Description-Listen
\usepackage{enumitem}
\setlist[description]{font=\rmfamily\mdseries\scshape\MakeLowercase}

%% Text nicht vertikal strecken
\raggedbottom

%% Hurenkinder verbieten
\makeatletter
  \widowpenalty=\@M
\makeatother

%% Akronyme
\newcommand{\acr}[1]{\textsc{\textls[40]{\MakeLowercase{#1}}}}

%% Verschiedene Sperrungen von small caps
\newcommand{\name}[1]{\textsc{\textls[40]{\MakeLowercase{#1}}}}
\newcommand{\tableHead}[1]{\textsc{\textls[60]{\MakeLowercase{#1}}}}

%% Etwas mehr Spacing für spaced small caps
\SetTracking[spacing={100*,166,}]{encoding=*,shape=ssc}{15}
\newcommand{\SAC}[1]{\textls{\MakeUppercase{#1}}}    % spaced all caps
\newcommand{\SLSC}[1]{\textssc{\MakeLowercase{#1}}}  % spaced lowercase small caps

%% Paragraph
%\setkomafont{paragraph}{\sscshape\mdseries}

%% Sections
\addtokomafont{section}{\normalsize\sscshape\lsstyle\mdseries\MakeLowercase}
\renewcommand*{\sectionformat}   {\upshape\mdseries\sscshape\thesection\enskip\hspace*{2px}}
\RedeclareSectionCommand[afterskip=\baselineskip]{section}

%% Subsections
\addtokomafont{subsection}{\normalsize\mdseries\itshape}
\renewcommand*{\subsectionformat}{\upshape\mdseries\scshape\thesubsection\enskip}
\RedeclareSectionCommand[afterskip=.75\baselineskip]{subsection}

%% Chapter
\RedeclareSectionCommand[beforeskip=0\baselineskip,afterskip=1\baselineskip,afterindent=false]{chapter}
\addtokomafont{chapter}{\mdseries\normalsize\lsstyle}
\renewcommand*{\chapterformat}{\thechapter}%\normalsize\textssc{kapitel \thechapter}}
\renewcommand{\chapterlinesformat}[3]{%
  \MakeUppercase{#3}\par%
  \marginnote{\smash{\fontsize{70}{60}\selectfont #2}}
  \rule[-.15\baselineskip]{\linewidth}{.5pt}\par\nobreak
}%

%% Satzspiegel
%\KOMAoptions{DIV=10}
\areaset{288pt}{480pt}%
\setlength{\marginparwidth}{8em}%
\setlength{\marginparsep}{1.5em}%

%% Keine Kopfzeilen
\pagestyle{plain}
 
%% Footnote-Design
\usepackage{footmisc}
\setlength{\footnotemargin}{0em}
\deffootnote{0em}{0em}{%
  \textsuperscript{\normalfont\thefootnotemark\ }}


%% Einzelne Zeichen
%\fboxsep0pt
%\newcommand{\Char}[1]{\colorbox{black!10}{\strut\textcolor{Maroon}{\hspace*{1px}#1\hspace*{1px}}}}
\newcommand{\Char}[1]{\textcolor{Maroon}{#1}}
\newcommand{\tab}[1]{\figureversion{tab}{#1}}

\usepackage[margin    =20pt,
            font      ={footnotesize},
            format    =plain,
            labelsep  =period]{caption}

%%% Local Variables:
%%% mode: latex
%%% TeX-master: "main"
%%% End:


\usepackage{menukeys}

\usepackage{lipsum}

%% Hyperref als letztes Paket
\usepackage[bookmarksnumbered=true,unicode,pdfusetitle,hidelinks]{hyperref}

\usepackage[nameinlink]{cleveref}

%% schräge Schrift („slanted“)
%% Quelle: https://tex.stackexchange.com/a/260952
\newsavebox\foobox
\newcommand{\slantbox}[2][.2]{\mbox{%
    \sbox{\foobox}{#2}%
    \hskip\wd\foobox
    \pdfsave
    \pdfsetmatrix{1 0 #1 1}%
    \llap{\usebox{\foobox}}%
    \pdfrestore
  }}
\def\textsl#1{\slantbox{#1}}

\begin{document}

\frontmatter
%% Titelseite
\begin{titlepage}
  \centering\large
  ~
  \vfill  
  \textcolor{Maroon}{\SAC{Typographie}} \\ \medskip
  \SLSC{Florian Kranhold \emph{\&} Philip Schwartz}
  \vfill
  {\fontsize{180}{60}\selectfont \textcolor{black!20}{\emph{ffi}}}
  \vfill
  Skript zum Kurs\\
  CdE-WinterAkademie 2024\,·\,25\\
  Windischleuba
  \vfill
\end{titlepage}

%% Kolophon
\thispagestyle{empty}%
~%
\vfill%
\noindent Dieses Skript wurde mit \LaTeX{} gesetzt. Als Schriftart wurde die
Minion von Robert Slimbach verwendet. Das Gesamtlayout ist inspiriert von Robert
Bringhursts \emph{The Elements of Typographic Style}.
\cleardoublepage

%%% Local Variables:
%%% mode: latex
%%% TeX-master: "main"
%%% End:


\chapter*{Vorwort}
In unserem Kurs --~und damit auch diesem Skript, an dem wir uns
orientieren~-- geht es um \emph{Typographie}: die Kunst (?), Texte
»schön« zu setzen.  Die Typographie dient vor allem einem Zweck,
nämlich den \emph{Inhalt von Texten gut erfassbar zu machen}.
Letztlich ordnet sich diesem Zweck alles unter.

Dabei gibt es viele »Regeln«, bei denen es sich vor allem um
historisch gewachsene Konventionen handelt: ein tatsächliches
\emph{richtig} oder \emph{falsch} gibt es in der Typographie selten.
Deshalb geht es uns vor allem darum, Euch im Kurs zu zeigen,
\emph{worauf} man typographisch überhaupt alles achten kann und was es
für \emph{Möglichkeiten} gibt~-- so erhaltet ihr eine Sprache, um
fundiert typographische Entscheidungen treffen zu können.  (An vielen
Stellen geben wir trotzdem Empfehlungen, deren Umsetzung ein
Schriftstück oft deutlich professioneller wirken lässt als ein
Abweichen davon.)

Wir (Florian und Philip) sind keine professionellen Typographen oder
Schriftsetzer; wir haben nur viel gefährliches Halbwissen\footnote{Und
  noch mehr Meinung!}, das wir gerne an Euch weitergeben möchten.
Insofern solltet ihr diesem Skript nicht unhinterfragt vertrauen~-- es
enthält vermutlich einige Ungenauigkeiten oder Fehler; an einigen
Stellen verkürzen wir auch stark.  Es gibt viel Fachliteratur zu
Typographie, die Ihr bei weitergehenden Fragen zu Rate ziehen könnt
und solltet.

Schließen möchten wir dieses Vorwort mit vielleicht der wichtigsten
(oder sogar einzigen?) Grundregel der Typographie: \emph{Konsistenz.}
Beim Erstellen von Texten muss man typographische Entscheidungen
treffen; sobald man das getan hat, sollte man diese dann auch
möglichst strikt umsetzen.  Nur so erreicht man für den jeweiligen
Text ein Gesamtbild, das Information klar wiedergibt.

Wir freuen uns auf einen schönen Kurs!

\begin{flushright}
  Troisdorf und Hannover,\\22. Dezember 2024
\end{flushright}

\mainmatter
\chapter{Zeichensetzung}

%% - Anführungszeichen (D/E)
%% - Striche (D/E)
%% - Leerzeichen

%%% Local Variables:
%%% mode: latex
%%% TeX-master: "main"
%%% End:

%% - Kursiv
%% - Unterstreichen (mist)
%% - Fett (modern)
%% - Kapitälchen
%% - Versalien
%% - Sperren von Kapitälchen und Versalien

%%% Local Variables:
%%% mode: latex
%%% TeX-master: "main"
%%% End:

\chapter{Schriften}
\label{ch:Schriften}

Das grundlegende Arbeitsmittel der Typographie ist natürlich die
\emph{Schrift}.  Mit ihr wollen wir uns in diesem Kapitel umfassend
beschäftigen.  Einige Aspekte haben wir schon gesehen; hier gehen wir
aber weiter ins Detail.  Da sich die Gestaltung von Schrift »im
Kleinen« abspielt (im Gegensatz zur Gestaltung von bspw. ganzen
Seiten), wird das entsprechende Teilgebiet der Typographie auch als
\emph{Mikrotypographie} bezeichnet.

\section{Begriffliches und Klassifikation}

Eine \emph{Glyphe} ist die konkrete graphische Realisierung eines
(abstrakten) Schriftzeichens.  So sind bspw. \Char{a},
\Char{\textlarger{a}}, \Char{\textsmaller{a}}, \Char{\textit{a}},
\Char{\textbf{a}}, \Char{\textsf{a}}, \Char{\textsf{\textit{a}}} alles
unterschiedliche Glyphen, die den \emph{lateinischen Kleinbuchstaben
  a} darstellen.

Zusammenhängende Texte werden in einer \emph{Schriftart} gesetzt; die
darin vorhandenen Glyphen sind alle ähnlich und zueinander passend
gestaltet~-- sonst ergibt sich {U\fontfamily{ptm}\selectfont
  n\fontfamily{pplj}\selectfont r\fontfamily{lmr}\selectfont
  u\fontfamily{uop}\selectfont h\fontfamily{pcr}\selectfont e}.  Liegt
eine Schriftart in verschiedenen Varianten vor, was üblicherweise der
Fall ist, so werden diese \emph{Schriftschnitte} genannt
(s. \cref{sec:Schnitte}).  Eine solche Schriftart mit mehreren
Schnitten nennt man dann auch eine \emph{Schriftfamilie}.\footnote{Das
  ist das Beste, was ich (Philip) nach längerer Recherche zur
  Unterscheidung der Begriffe \emph{Schriftart} und \emph{-familie}
  herausbekommen zu haben glaube.  Allerdings fällt mir spontan
  (während ich diesen Text schreibe) kein Beispiel für eine Schriftart
  ein, von der es nur einen Schnitt gibt, die also keine Familie ist.}

\begin{itemize}
\item Schriftarten nach vielen Dingen klassifizierbar.  Alle
  Schriften, die uns hier interessieren, sind \emph{Antiqua}-Schriften
  (im Gegensatz zu gebrochenen Schriften, d.\,h. Fraktur und so).
\item Gibt diverse Klassen (s. bspw. \acr{DIN}~16518).  Für uns hier
  im Skript relevant: Mit Serifen und ohne (= Grotesk = Sans Serif =
  Serifenlose Linear-Antiqua).
\end{itemize}

\section{Schriftschnitte}
\label{sec:Schnitte}

\section{Liniensystem}
\label{sec:Linien}

%% Inkl. Maßeinheiten und so, und Zeilenabstand

\section{Kerning und Spacing}
\label{sec:KernSpace}

%% Hier microtype erwähnen?

\section{Ligaturen}

%% Ligaturbrecher erwähnen!

\section{Ziffern}
\label{sec:Ziffern}

%% Minuskelziffern, Majuskelziffern, Tabellenziffern / lining figures

\section{Beispiele}

{\fontfamily{pplj}\selectfont Palatino 1234 \emph{kursiv} \textsc{kapitälchen}}
%% Garamond, Palatino, Minion, Optima/Classico, Times, Helvetica

%% Jeweils auch die lustigen Namen nach DIN 16518 erwähnen!

%%% Local Variables:
%%% mode: latex
%%% TeX-master: "main"
%%% End:

\chapter{Seitenlayout}

%% - Satzspiegel/Zeilenlänge
%% - Zeilenabstand
%% - Seitenumbrüche
%% - Schriftgrößen mischen
%% - Kopf-/Fußzeilen/Pagination
%% - andere Formate (z.B. Plakate, Titelseiten, …)

%%% Local Variables:
%%% mode: latex
%%% TeX-master: "main"
%%% End:


\end{document}

%%% Local Variables:
%%% mode: latex
%%% TeX-master: t
%%% End:
