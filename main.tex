\documentclass[british,ngerman]{scrbook}

\author{Florian Kranhold, Philip Schwartz}
\title {Typographie}

\KOMAoptions{
  numbers=noenddot,
  headings=standardclasses,
  chapterprefix=false,
  paper=a5,                   % Papierformat ist DIN B5
  fontsize=11pt
}

%% Encoding
\usepackage[T1]{fontenc}
\usepackage[utf8]{inputenc}

%% Hierfür wird das CTAN-Paket MinionPro, zusammen mit den
%% Type-1-Dateien der Minion Pro, benötigt.
\usepackage[fullfamily,textosf,mathlf,opticals,footnotefigures]{MinionPro}

%% Serifenlose
\usepackage{classico}
%% Damit die (Fake-)Kapitälchen auch mit neuen Versionen des
%% classico-Pakets funktionieren
\pdfmapfile{=classico_uop.map}
\makeatletter
%% Leider können wir nicht einfach zusätzliche font shapes definieren,
%% sondern müssen alles hier tun.  Nun gut.
\DeclareFontFamily{T1}{URWClassico-LF}{}
\DeclareFontShape{T1}{URWClassico-LF}{b}{it}{<-> URWClassico-BoldItalic-lf-t1}{}
\DeclareFontShape{T1}{URWClassico-LF}{bold}{it}{<-> alias * URWClassico-LF/b/it}{}
\DeclareFontShape{T1}{URWClassico-LF}{b}{n}{<-> URWClassico-Bold-lf-t1}{}
\DeclareFontShape{T1}{URWClassico-LF}{bold}{n}{<-> alias * URWClassico-LF/b/n}{}
\DeclareFontShape{T1}{URWClassico-LF}{b}{sl}{<-> ssub * URWClassico-LF/b/it}{}
\DeclareFontShape{T1}{URWClassico-LF}{bold}{sl}{<-> ssub * URWClassico-LF/bold/it}{}
\DeclareFontShape{T1}{URWClassico-LF}{b}{sc}{<-> uopbc8t}{}
\DeclareFontShape{T1}{URWClassico-LF}{bold}{sc}{<-> alias * URWClassico-LF/b/sc}{}
\DeclareFontShape{T1}{URWClassico-LF}{m}{it}{<-> URWClassico-Italic-lf-t1}{}
\DeclareFontShape{T1}{URWClassico-LF}{regular}{it}{<-> alias * URWClassico-LF/m/it}{}
\DeclareFontShape{T1}{URWClassico-LF}{m}{n}{<-> URWClassico-Regular-lf-t1}{}
\DeclareFontShape{T1}{URWClassico-LF}{regular}{n}{<-> alias * URWClassico-LF/m/n}{}
\DeclareFontShape{T1}{URWClassico-LF}{m}{sl}{<-> ssub * URWClassico-LF/m/it}{}
\DeclareFontShape{T1}{URWClassico-LF}{regular}{sl}{<-> ssub * URWClassico-LF/regular/it}{}
\DeclareFontShape{T1}{URWClassico-LF}{m}{sc}{<-> uoprc8t}{}
\DeclareFontShape{T1}{URWClassico-LF}{regular}{sc}{<-> alias * URWClassico-LF/regular/sc}{}
\DeclareFontShape{T1}{URWClassico-LF}{bx}{it}{<-> ssub * URWClassico-LF/b/it}{}
\DeclareFontShape{T1}{URWClassico-LF}{bx}{sl}{<-> ssub * URWClassico-LF/b/sl}{}
\DeclareFontShape{T1}{URWClassico-LF}{bx}{n}{<-> ssub * URWClassico-LF/b/n}{}
\DeclareFontShape{T1}{URWClassico-LF}{bx}{sc}{<-> ssub * URWClassico-LF/b/sc}{}
\makeatother

%% Eine Dicktengleiche
\usepackage[scaled=.8]{beramono}

%% Sprache, Anführungszeichen, Datum
\usepackage{babel}
\usepackage{csquotes}
\usepackage[cleanlook]{isodate}

%% Mikrotypographie
\usepackage{ellipsis}
\usepackage[babel,letterspace=170]{microtype}

%% Fancy tables
\usepackage{booktabs}

%% Farben
\usepackage[dvipsnames]{xcolor}

\usepackage[size=tiny]{todonotes}

%% Wird für die Kapitelziffern benötigt.
\usepackage{marginnote}

%% Sinnvoll konfigurierbare Listen
\usepackage{enumitem}
\setlist{nosep}

%% Text nicht vertikal strecken
%\raggedbottom

%% Hurenkinder verbieten
\makeatletter
  \widowpenalty=\@M
\makeatother

%% Wir wollen, dass auch, wenn wir für Typographie alles groß oder
%% klein setzen, trotzdem die korrekten Buchstaben kopiert werden und
%% in PDF-Metadaten landen.
\usepackage{accsupp}
\makeatletter
\newcommand*{\c@pyCorrectly}[2]{\texorpdfstring{%
    \protect\BeginAccSupp{%
      method=pdfstringdef,%
      ActualText=#1%
    }%
    #2%
    \protect\EndAccSupp{}}%
  {#1}%
}

%% Akronyme
%% Optionales Argument für Reintext-Version (bspw. ohne Kerning)
\usepackage{xifthen}

\newcommand*{\@cr}[2]{\c@pyCorrectly{#1}{\textsc{\textls[40]{\MakeLowercase{#2}}}}}
\newcommand*{\acr}[2][]{\ifthenelse{\isempty{#1}}{\@cr{#2}{#2}}{\@cr{#1}{#2}}}

%% Verschiedene Sperrungen von small caps
\newcommand*{\name}[1]{\c@pyCorrectly{#1}{\textsc{\textls[40]{\MakeLowercase{#1}}}}}
\newcommand*{\tableHead}[1]{\c@pyCorrectly{#1}{\textsc{\textls[60]{\MakeLowercase{#1}}}}}

%% Etwas mehr Spacing für spaced small caps
\SetTracking[spacing={100*,166,}]{encoding=*,shape=ssc}{15}
\newcommand*{\SAC}[1]{\c@pyCorrectly{#1}{\textls{\MakeUppercase{#1}}}}    % spaced all caps
\newcommand*{\SLSC}[1]{\c@pyCorrectly{#1}{\textssc{\MakeLowercase{#1}}}}  % spaced lowercase small caps
\makeatother

%% Paragraph
%\setkomafont{paragraph}{\sscshape\mdseries}

%% Sections
\addtokomafont{section}{\normalsize\sscshape\lsstyle\mdseries\MakeLowercase}
\renewcommand*{\sectionformat}   {\upshape\mdseries\sscshape\thesection\enskip\hspace*{2px}}
\RedeclareSectionCommand[afterskip=1\baselineskip plus 2pt minus 2pt]{section}

%% Subsections
\addtokomafont{subsection}{\normalsize\mdseries\itshape}
\renewcommand*{\subsectionformat}{\upshape\mdseries\scshape\thesubsection\enskip}
\RedeclareSectionCommand[afterskip=.75\baselineskip plus 2pt minus 2pt]{subsection}

%% Chapter
\RedeclareSectionCommand[beforeskip=0\baselineskip,afterskip=1\baselineskip,afterindent=false]{chapter}
\addtokomafont{chapter}{\mdseries\normalsize\lsstyle}
\renewcommand*{\chapterformat}{\thechapter}%\normalsize\textssc{kapitel \thechapter}}
\renewcommand{\chapterlinesformat}[3]{%
  \MakeUppercase{#3}\par%
  \marginnote{\smash{\fontsize{70}{60}\selectfont #2}}
  \rule[-.15\baselineskip]{\linewidth}{.5pt}\par\nobreak
}%

%% Satzspiegel
%\KOMAoptions{DIV=10}
\areaset{288pt}{480pt}%
\setlength{\marginparwidth}{8em}%
\setlength{\marginparsep}{1.5em}%
\setlength{\footskip}{2.5\baselineskip}

%% Keine Kopfzeilen
\pagestyle{plain}
 
%% Footnote-Design
\usepackage{footmisc}
\setlength{\footnotemargin}{0em}
\deffootnote{0em}{0em}{%
  \textsuperscript{\normalfont\thefootnotemark\ }}


%% Einzelne Zeichen
%\fboxsep0pt
%\newcommand{\Char}[1]{\colorbox{black!10}{\strut\textcolor{Maroon}{\hspace*{1px}#1\hspace*{1px}}}}
\newcommand*{\Char}[1]{\textcolor{Maroon}{#1}}
\newcommand*{\tab}[1]{\figureversion{tab}{#1}}

\usepackage[margin    =20pt,
            font      ={footnotesize},
            format    =plain,
            labelsep  =period]{caption}

%%% Local Variables:
%%% mode: latex
%%% TeX-master: "main"
%%% End:


\newcommand*{\codepoint}[1]{\texttt{U+\MakeUppercase{#1}}}

\usepackage{menukeys}

\usepackage{lipsum}

%% Hyperref als letztes Paket
\usepackage[bookmarksnumbered=true,unicode,pdfusetitle,hidelinks]{hyperref}

\usepackage[nameinlink]{cleveref}

%% schräge Schrift („slanted“)
%% Quelle: https://tex.stackexchange.com/a/260952
\newsavebox\foobox
\newcommand{\slantbox}[2][.2]{\mbox{%
    \sbox{\foobox}{#2}%
    \hskip\wd\foobox
    \pdfsave
    \pdfsetmatrix{1 0 #1 1}%
    \llap{\usebox{\foobox}}%
    \pdfrestore
  }}
\def\textsl#1{\slantbox{#1}}

\begin{document}

\frontmatter
%% Titelseite
\begin{titlepage}
  \centering\large
  ~
  \vfill  
  \textcolor{Maroon}{\SAC{Typographie}} \\ \medskip
  \SLSC{Florian Kranhold \emph{\&} Philip Schwartz}
  \vfill
  {\fontsize{180}{60}\selectfont \textcolor{black!20}{\emph{ffi}}}
  \vfill
  Skript zum Kurs\\
  CdE-WinterAkademie 2024\,·\,25\\
  Windischleuba
  \vfill
\end{titlepage}

%% Kolophon
\thispagestyle{empty}%
~%
\vfill%
\noindent Dieses Skript wurde mit \LaTeX{} gesetzt. Als Hauptschrift
wurde die Minion von Robert Slimbach verwendet. Das Gesamtlayout ist
\mbox{inspiriert} von Robert Bringhursts \emph{The Elements of
  Typographic Style}.
\cleardoublepage

%%% Local Variables:
%%% mode: latex
%%% TeX-master: "main"
%%% End:


\chapter*{Vorwort}
In unserem Kurs --~und damit auch diesem Skript, an dem wir uns
orientieren~-- geht es um \emph{Typographie}: die Kunst (?), Texte
»schön« zu setzen.  Die Typographie dient vor allem einem Zweck,
nämlich den \emph{Inhalt von Texten gut erfassbar zu machen}.
Letztlich ordnet sich diesem Zweck alles unter.

Dabei gibt es viele »Regeln«, bei denen es sich vor allem um
historisch gewachsene Konventionen handelt: ein tatsächliches
\emph{richtig} oder \emph{falsch} gibt es in der Typographie selten.
Deshalb geht es uns vor allem darum, Euch im Kurs zu zeigen,
\emph{worauf} man typographisch überhaupt alles achten kann und was es
für \emph{Möglichkeiten} gibt~-- so erhaltet Ihr eine Sprache, um
fundiert typographische Entscheidungen treffen zu können.  (An vielen
Stellen geben wir trotzdem Empfehlungen, deren Umsetzung ein
Schriftstück oft deutlich professioneller wirken lässt als ein
Abweichen davon.)

Wir (Florian und Philip) sind keine professionellen Typographen oder
Schriftsetzer; wir haben nur viel gefährliches Halbwissen\footnote{Und
  noch mehr Meinung!}, das wir gerne an Euch weitergeben möchten.
Insofern solltet Ihr diesem Skript nicht unhinterfragt vertrauen~-- es
enthält vermutlich einige Ungenauigkeiten oder Fehler; an einigen
Stellen verkürzen wir auch stark.  Es gibt viel Fachliteratur zu
Typographie, die Ihr bei weitergehenden Fragen zu Rate ziehen könnt
und solltet.

Schließen möchten wir dieses Vorwort mit vielleicht der wichtigsten
(oder sogar einzigen?) Grundregel der Typographie: \emph{Konsistenz.}
Beim Erstellen von Texten muss man typographische Entscheidungen
treffen; sobald man das getan hat, sollte man diese dann auch
möglichst strikt umsetzen.  Nur so erreicht man für den jeweiligen
Text ein Gesamtbild, das Information klar wiedergibt.

Wir freuen uns auf einen schönen Kurs!

\begin{flushright}
  Troisdorf und Hannover,\\22. Dezember 2024
\end{flushright}

\mainmatter
\chapter{Zeichensetzung}

Dieses erste Kapitel beschäftigt sich streng genommen noch nicht mit
Typographie, sondern lediglich mit der semantisch korrekten Verwendung von
Sonderzeichen in deutschen und englischen Texten. Für diese Problematik genügt
es also, einen Text als reine Kette von wohldefinierten Zeichen zu betrachten.
Durch den Unicode-Standard ist jedes dieser Zeichen durch einen
eindeutigen hexadezimalen Codepoint (z.\,B. \texttt{0061} für »a«) festgelegt.

Die korrekte Verwendung grundlegender Satzzeichen (Punkt, Komma und Semikolon sowie
Frage- und Ausrufezeichen) setzen wir als bekannt voraus. Bei einigen
anderen Sonderzeichen gibt es jedoch Subtilitäten, die oft missachtet werden.

\section{Anführungszeichen}

Die Standardversionen der \emph{Anführungszeichen} sowohl im Deutschen als auch
im Englischen sehen aus wie (gedrehte und verschobene) Kommata, also
\Char{„}\,\Char{“}\,\Char{”} sowie die einfachen Varianten
\Char{‚}\,\Char{‘}\,\Char{’}.  Je nachdem, wie sie gedreht sind, werden die
einzelnen Striche als »6en« oder »9en« bezeichnet -- denn sie sehen
aus wie verkleinerte und ausgefüllte Versionen dieser Ziffern (ein Komma
\Char{,} sieht wie eine 9 aus).\looseness-1

Im Deutschen wird die Anführungszeichenkombination »99–66« benutzt, wobei die
öffenen Anführungszeichen unten, die schließenden jedoch oben stehen, zum
Beispiel: „Hallo!“ Bei verschachtelten Anführungszeichen werden innen einfache
Anführungszeichen, also »9–6«, benutzt: „Er sagte: ‚Hallo!{‘}, und ging fort.“
Wir merken an, dass das einfache öffnende deutsche Anführungszeichen \Char{‚}
(\texttt{201a}) einen anderen Unicode-Codepoint als das Komma \Char{,}
(\texttt{002c}) besitzt.\looseness-1

Im Englischen wird (wenigstens im Britischen) die Kombination »6–9«,
beide oben, verwendet, zum Beispiel: \foreignlanguage{british}{‘Hi!’}
Bei verschachtelten Anführungszeichen werden innen doppelte
Anführungszeichen, also »66–99«, benutzt:
\foreignlanguage{british}{‘“Hi!”, he said, and went away.’}

Alternativ können im Deutschen auch die sogenannten \emph{Guillemets}
\Char{»}\,\Char{«} verwendet werden, so wie in diesem Skript.  Innen
werden dann die einfachen Guillements \Char{›}\,\Char{‹} verwendet.

Wir weisen darauf hin, dass die auf einer Tastatur leichter zu erreichenden
\emph{Ersatzzeichen} \Char{\char"22} und \Char{\textquotesingle} (bei deutscher
\acr{QWERTZ}-Belegung \keys{\shift+2} bzw. \keys{\shift+\#}) Überbleibsel
aus der Schreibmaschinenzeit und semantisch falsch sind. Die freistehenden
Akzent\-zeichen \Char{\textasciigrave} und \Char{´} sind ebenfalls keine
Anführungszeichen.

Viele Programme (wie etwa \acr{MS} Word) ersetzen Ersatzzeichen automatisch durch
Anführungszeichen in der eingestellten Sprache. In anderen Situationen ist es
jedoch notwendig, die Zeichen manuell einzugeben. Auf einer Linux-Tastatur mit
deutschen Layout gibt es hierfür praktische Tastenkombinationen; unter Windows
kann man bei gehaltener \keys{Alt}-Taste nach einem \keys{{+}} den entsprechenden
Unicode-Codepoint eingeben und anschließend die \keys{Alt}-Taste
lösen. In \cref{tab:quotationMarks} finden sich die entsprechenden
Tastenkombinationen und Codepoints.

\begin{table}
  \centering
  \begin{tabular}{clc}
    \toprule
    \tableHead{Zeichen} & \tableHead{Linux-Tastatur} & \tableHead{Code}\\
    \midrule
    \Char{„} & \keys{\AltGr+v} & \texttt{201e}\\
    \Char{“} & \keys{\AltGr+b} & \texttt{201c}\\
    \Char{”} & \keys{\AltGr+n} & \texttt{201d}\\
    \Char{‚} & \keys{\AltGr+\shift+v} & \texttt{201a}\\
    \Char{‘} & \keys{\AltGr+\shift+b} & \texttt{2018}\\
    \Char{’} & \keys{\AltGr+\shift+n} & \texttt{2019}\\
    \Char{»} & \keys{\AltGr+y} & \texttt{00bb}\\
    \Char{«} & \keys{\AltGr+x} & \texttt{00ab}\\
    \Char{›} & \keys{\AltGr+\shift+y} & \texttt{203a}\\
    \Char{‹} & \keys{\AltGr+\shift+x} & \texttt{2039}\\
    \bottomrule
  \end{tabular}
  \caption{Anführungszeichen und wie sie erzeugt werden können.}\label{tab:quotationMarks}
\end{table}

\section{Apostrophe}

Der \emph{Apostroph} \Char{’} sieht genauso aus wie ein einfaches schließendes
englisches Anführungszeichen. Auch wenn es sich semantisch von diesem
unterscheidet, hat es den gleichen Codepoint und wird daher genauso erzeugt.
Man sollte die Sprache, in der man schreibt, gut genug beherrschen, um zu
wissen, wo Apostrophe hinkommen und wo nicht.  Wichtig: Die freistehenden
Akzentzeichen \Char{\textasciigrave} und \Char{´} sind, genau wie das
Ersatzzeichen \Char{\textquotesingle}, \emph{keine} Apostrophe!\looseness-1

\section{Striche}

Es gibt verschiedene Arten von (horizontalen) \emph{Strichen}, die in Texten
aller Art auftauchen.  Hier eine kleine Auswahl:
\Char{-}\,\Char{--}\,\Char{---}\,\Char{$-$}\,\Char{$=$}.  Diese Striche haben
verschiedene Aufgaben, die gerne verwechselt werden.  Typographisch
unterscheiden wir die Striche nach ihrer Länge.

Diese Längen werden als Anteil an einem \emph{Geviert} angegeben~--
eine Einheit, die auf die Zeit des Bleisatzes zurückgeht.  Uns soll
hier folgende Faustregel genügen: Ein Geviert ist, je nach Schriftart,
etwa das \smallfrac43-fache der Höhe eines Großbuchstabens
(vgl. \cref{sec:Linien}).

\subsection{Viertelgeviertstrich}
Der Viertelgeviertstrich \Char{-} ist der kürzeste aller üblichen
Striche.  Er wird auf Deutsch sowie auf Englisch als \emph{Trennstrich} bei
Worttrennungen an Zeilenumbrüchen verwendet.  Auf Deutsch ist er außerdem der
\emph{Bindestrich} in nicht zusammenge\-schriebenen Komposita, bspw. in
»Anti-Terror-Anschlag«.  Dieses grammatikalische Phänomen nennt man
\emph{Durchkopplung}; es betrifft auch die Zusammensetzung von Eigennamen aus
mehreren Wörtern mit weiteren Wörtern zu einem Kompositum: bspw. wird aus
»Johann Sebastian Bach« und »Straße« die »Johann-Sebastian-Bach-Straße«.

Im Englischen gibt es üblicherweise \emph{keine} Durchkopplung, außer
bei zusammengesetzten Adjektiven: es heißt »apple tree«, »anti-terror
attack« oder »Johann Sebastian Bach Street«.

Im Deutschen werden auch Dinge, die nach mehreren Personen benannt
sind, als normales Kompositum betrachtet und dementsprechend mit
Bindestrich gesetzt.  Dieses Skript könnte man bspw. das
»Kranhold-Schwartz-Skript« nennen.  (Auf Englisch ist das anders,
siehe unten.)

\subsection{Halbgeviertstrich}
\label{subsec:halbgeviert}
Der Halbgeviertstrich \Char{--} ist doppelt so lang wie der
Viertelgeviertstrich.  Er ist der \emph{Bis-Strich} in Zeitraumangaben
wie »in den Jahren 2012--2023« oder »geöffnet Mo.--Fr. 9--18 Uhr«.
Dabei ist zu beachten, dass er \emph{nicht} in der Phrase »von … bis
…« das »bis« ersetzt: »von 11--13 Uhr« ist falsch.  Auch in
Streckenangaben steht ein Halbgeviertstrich: »Auf der Strecke
Köln--Karlsruhe fährt der \acr{ICE} 203.«

Er kann auch als Aufzählungszeichen verwendet werden (dann nennt man
ihn manchmal »Spiegelstrich«):
 \begin{itemize}[label=--,nosep]
 \item Es gibt sone und
 \item solche, und
 \item dann gibt's noch ganz andere!
 \end{itemize}
 Meistens sieht das aber nicht so gut aus und Aufzählungs\emph{punkte}
 \Char{\textbullet} sind besser.

Im Deutschen ist der Halbgeviertstrich, umschlossen von Leerzeichen,
auch der \emph{Gedankenstrich}: »Ein Gedankenstrich sieht ganz gut
aus~-- manchmal zumindest.«  Dabei ist darauf zu achten, wie sich die
Striche an Zeilenumbrüchen verhalten: Wenn ein Gedankenstrich einen
Satz in zwei Teile teilt, soll höchstens \emph{nach} ihm umgebrochen
werden (er »klebt« also am ersten Satzteil); wenn Gedankenstriche
einen Einschub umschließen, soll höchstens \emph{vor} dem vorderen und
\emph{nach} dem hinteren umgebrochen werden (sie »kleben« also am
Einschub).  Das sieht dann also wie folgt aus:  Manche --~Dinge
umschließende~-- Striche kleben an Wörtern, wohingegen~--
aber Stopp, das führt jetzt zu weit.

Auf Englisch wird traditionell der Halbgeviertstrich benutzt, um nach
mehreren Personen benannte Dinge zu schreiben.  Dieses Skript wären
dann also bspw. die »\foreignlanguage{british}{Kranhold--Schwartz
  course notes}«.  Manchmal (insb. in \acr{US}-amerikanischen /
nordamerikanischen Kontexten, aber nicht nur da) wird dafür
stattdessen aber wie auf Deutsch ein Viertelgeviertstrich verwendet.

\subsection{Geviertstrich}

Der Geviertstrich \Char{---} wird im Deutschen üblicherweise
nicht verwendet; eine Ausnahme stellen Quellenangaben bei
Zitaten dar:
\begin{displayquote}
  Musik ist wie ein Hamburger,\\
  die Noten sind die \mbox{Gurken}.

  \quad---~J.\,S.\,Bach
\end{displayquote}
%
In englischer Typographie wird oft ein Geviertstrich ohne umschließende
Leerzeichen als Gedankenstrich verwendet: »\foreignlanguage{british}{This looks
  quite interesting---but also somewhat peculiar for Germans.}« Oft wird aber
auch im Englischen ein Gedankenstrich durch einen Halbgeviertstrich mit
Leerzeichen realisiert.

Der Viertelgeviertstrich hat auf einer \acr{QWERTZ}-Tastatur eine eigene Taste.
In \cref{tab:striche} steht, wie man Halbgeviert- und Geviertstriche auf einer
Tastatur erzeugen kann.

\begin{table}
  \centering
  \begin{tabular}{lclc}
    \toprule
    \tableHead{Zeichen} & & \tableHead{Linux-Tastatur} & \tableHead{Code}\\
    \midrule
%    \midrule
    Halbgeviertstrich & \Char{–} & \keys{\AltGr+-} & \texttt{2013}\\
    Geviertstrich & \Char{—} & \keys{\AltGr+\shift+-} & \texttt{2014}\\
%    Minuszeichen & \Char{\textminus} & & \texttt{2212}\\
    \bottomrule
  \end{tabular}
  \caption{Striche und wie sie erzeugt werden können.}\label{tab:striche}
\end{table}

\subsection{Minus- und Gleichheitszeichen}

Das im Formelsatz verwendete \emph{Minuszeichen} \Char{$-$} ist semantisch
verschieden von den vorher beschriebenen Text-Strichen; in den meisten
Schriftarten sieht es auch anders aus.\footnote{Im Formelsatz verwendet \LaTeX\ 
  \emph{nochmal} andere Zeichen: Das korrekte Minus\-zeichen in der Schriftart
  dieses Dokuments ist \Char{\textminus}; das des Formelsatzes ist \Char{$–$}.}
Seine Breite entspricht der des Pluszeichens, und es ist vertikal zentriert:
$1+2-3=0$.

Auch das \emph{Gleichheitszeichen} \Char{$=$} besteht natürlich aus horizontalen
Strichen~-- nämlich \emph{Zwillings-Linien}, wie sein Erfinder
\name{\textsc{robert recorde} (ca. 1510--1558)} schrieb:
\begin{displayquote}
  \foreignlanguage{british}{And to avoide the tediouse repetition of
    these woordes : is equalle to : I will sette as I doe often in
    woorke use, a paire of paralleles, or Gemowe lines of one lengthe,
    thus: $=$, bicause noe .2. thynges, can be moare equalle.}

  \quad---~Robert Recorde, \emph{The Whetstone of Witte} (1557)
\end{displayquote}

% In gebrochenen Schriften sieht außerdem der Trennstrich einem
% Gleichheitszeichen ähnlich.  Wir beschäftigen uns hier aber nicht mit
% gebrochenen Schriften.
% Der Viertelgeviertstrich hat auf einer \acr{QWERTZ}-Tastatur eine eigene Taste;
% das Gleichheitszeichen kann mit \keys{\shift+0} erzeugt werden. Wie die übrigen
% Striche erzeugt werden, steht in \cref{tab:striche}. Will man explizit das
% Minuszeichen als UnicodeFür das Minuszeichen gibt es leider keine praktische
% Tastenkombination unter Linux; in einigen Desktopumgebungen (wie etwa
% \acr{GNOME}) kann nach \keys{Strg+\shift+u} der Codepoint eingegeben werden, der
% danach mit \keys{\return} bestätigt werden muss.  In \LaTeX{} taucht das
% Minuszeichen ohnehin zumeist in der Mathematikumgebung auf, in einer solchen
% wird es mit einem Viertelgeviertstrich codiert.



\section{Leerzeichen}

Das gewöhnliche Leerzeichen (\texttt{0020}) hat üblicherweise die Breite eines
Viertelgevierts. Man sollte darauf achten, Leerzeichendopplungen zu vermeiden;
solche fallen oft erst beim genauen Gegenlesen auf. (In \LaTeX\ oder \acr{HTML}
spielen sie zum Glück keine Rolle, weil sie beim Kompilieren bzw. Rendern
ignoriert werden.)

Möchte man (wie z.\,B. in \cref{subsec:halbgeviert}) verhindern, dass bei einem
ganz konkreten Leerzeichen ein Zeilenumbruch erfolgt, so verwendet man an dieser
Stelle ein \emph{(umbruch-)geschütztes Leerzeichen} gleicher Breite
(\texttt{00a0}). In \LaTeX\ wird dies durch \verb!~! realisiert.

Das \emph{schmale Leerzeichen} (\texttt{2009}) ist zwischen \smallfrac18\ und
\smallfrac16\ eines Gevierts breit und stets geschützt. In \LaTeX\ kann
es durch \verb!\,! erzeugt werden. Es kommt u.\,a. an folgenden Stellen zum Einsatz:
\begin{itemize}[nosep]
\item Abkürzungen wie etwa d.\,h., z.\,B. oder i.\,d.\,R.,
\item Datumsangaben wie etwa 28.\,12.\,2024,
\item Abkürzungen mit Nummern wie etwa §\,3, Thm.\,4.7,
\item Zahlen mit Einheitenzeichen wie etwa 87\,km oder 37\,\%.
\end{itemize}

%% Doppelte Leerzeichen vermeiden
%% Schmale Leerzeichen
%% Englisch?
%% Umbruchgeschütz

\section{Auslassungspunkte}

Der Dreipunkt \Char{…} ist nicht etwa die Aneinanderreihung von
drei (oder noch schlimmer: 4 oder 17) Punkten \Char{{.}{.}{.}}, sondern ein eigenständiges
Zeichen (\texttt{2026}, Linux-Tastatur \keys{\AltGr+.}).
%%% Local Variables:
%%% mode: latex
%%% TeX-master: "main"
%%% End:

\chapter{Hervorhebungen}
\label{ch:Emph}

Oft ist man in der Situation, dass der eigentliche Text in einer bestimmten
Schrift gesetzt ist, aber beispielsweise einzelne Wörter oder
Gliederungselemente (wie Titel und Überschriften) zur besseren Übersichtlichkeit
\emph{hervorgehoben} werden sollen. Hierfür gibt es verschiedene Möglichkeiten, die
je nach Situation verschieden geeignet sein können.

\section{Kursivsatz}

Die\marginnote{\texttt{\textbackslash textit}} wohl gängigste Methode, Passagen
im Fließtext hervorzuheben, ist, sie \emph{kursiv} zu setzen. Diese Hervorhebung
ist dezent und seit vielen Jahrhunderten etabliert. Fast alle Schriftarten
bringen eigene Schriftschnitte für kursiven Text (siehe \cref{sec:Schnitte})
mit. Sollte ein kursiver Schriftschnitt fehlen, könnte das Textsatzprogramm
versuchen, den Kursivsatz künstlich durch \textsl{Schrägen} zu erzeugen; davon
ist abzuraten.

\section{Fettdruck}

Eine\marginnote{\texttt{\textbackslash textbf}} heutzutage sehr verbreitete,
aber noch gar nicht so alte Methode der Hervorhebung ist die Verwendung von
{\fontseries{eb}\selectfont {Fett-}} oder besser
\textbf{Mittelfettbuchstaben}. Man sollte bedenken, dass diese Auszeichnungsart
sehr hervorsticht. Üblicher ist es, Gliederungselemente wie Überschriften mit
fetten Buchstaben zu setzen. In einem klassischen Layout (wie dem dieses
Skripts) taucht diese Auszeichnung kaum auf. Eine
\emph{{\fontseries{b}\selectfont{Kombination}}} von fett und kursiv ist zwar
möglich, sticht aber sehr hervor und sollte deshalb mit Vorsicht eingesetzt
werden.\looseness-1

\section{Kapitälchen}

Viele\marginnote{\texttt{\textbackslash textsc}} Schriftarten besitzen eine
Variante der Großbuchstaben, die so hoch wie übliche Kleinbuchstaben ist. Diese
Glyphen heißen \textsc{kapitälchen}. Man bemerke, dass sich diese Zeichen
deutlich von einer reinen Skalierung der Großbuchstaben unterscheiden, zum
Vergleich: \textsmaller[2]{KAPITÄLCHEN}; der wohl größte Unterschied ist die bei
den skalierten Großbuchstaben deutlich zu geringe Strichdicke. Bevor man
Kapitälchen verwendet, sollte man also prüfen, ob die eingesetzte Schriftart
solche enthält, da man sonst Gefahr läuft, Kapitälchen durch Skalierung zu
imitieren.\looseness-1

Bei Kapitälchen wird nicht zwischen Groß- und Kleinbuchstaben
unterschieden. Dennoch geschieht es oft, dass zur Unterscheidung »normale«
Großbuchstaben in Kombination mit Kapitälchen gesetzt werden, also
\textsc{Kapitälchen}. Gerade hier ist es wichtig, »echte« Kapitälchen zu
nutzen; ansonsten erhält man so etwas wie K\textsmaller[2]{APITÄLCHEN}, also
unterschiedliche Strichdicken im selben Wort.

\marginnote{\texttt{\textbackslash textls}}Kapitälchen werden häufig
\emph{gesperrt}, d.\,h. der Abstand zwischen den einzelnen Buchstaben wird
künstlich vergrößert, zum Vergleich:
\textls[50]{\textsc{kapitälchen}}. Kleinbuchstaben will man hingegen nicht
\textls[50]{sperren}.

Kapitälchen können an verschiedenen Stellen zum Einsatz kommen, und je nach
Anwendung bietet sich ein anderer Sperrfaktor an. In diesem Skript werden
Kapitälchen unter anderem für die Überschriften der Abschnitte verwendet. Nicht
selten werden Namen wie \name{Immanuel Kant} (1724–1804) mit
Kapitälchen hervorgehoben.\looseness-1

\section{Exkurs: Akronyme}

Es ist ratsam, \emph{Akronyme} wie etwa \acr{spd} oder \acr{PHP} mit leicht
gesperrten Kapitälchen zu setzen, damit sie nicht, wie etwa SPD und PHP,
hervorstechen. Bei Akronymen, die Zahlen enthalten (wie etwa \acr{3D},
\acr{C-\kern-1pt3PO} oder \acr{BW\kern-.3px V\,29}), müssen dann aber
\emph{Minuskelziffern} (siehe \cref{sec:Ziffern}) verwendet werden, sonst erhält
man \acr{\figureversion{lf}{3}d}, \acr{C-\kern-1pt\figureversion{lf}{3}PO} und
\acr{BW\kern-.3px V\,\figureversion{lf}{29}}.

\section{Versalien}

Die einfachste, aber deutlichste (oder besser: kompromissloseste) Art
der Hervorhebung ist die Verwendung von \textls[60]{VERSALIEN} (also
Großbuchstaben), die dann gesperrt werden sollten. Wir verwenden sie für
Kapitelüberschriften und den Skripttitel.  Diese Auszeichnungsform ist etwas aus
der Mode gekommen, findet sich aber in alten Texten sehr häufig.\looseness-1

\section{Unterstreichungen}

Unterstreichungen sind \underline{Mist}, vor allem, wenn Buchstaben vorkommen,
die \underline{Unterlänge} haben. Mogelt man und ignoriert Kollisionen, so wird
es auch nicht besser: \underline{\smash{Unterlänge}}.

\section{Umsetzung in \LaTeX}

Wenn einzelne Textpassagen auf eine der oben beschriebenen Weisen hervorgehoben
werden sollen, können hierfür die am Seitenrand vermerkten Befehle genutzt
werden, etwa: \verb!\textit{Hi}!.  Es ist allerdings zu empfehlen, eine
bevorzugte Hervorhebungsmethode festzulegen (standardmäßig: Kursivsatz) und dann
den Befehl \verb!\emph!  (für »emphasize«) zu verwenden.

Der Befehl \verb!\textls! ist Teil des \texttt{microtype}-Pakets.  Über einen optionalen
Parameter (oder global zu Beginn) kann festgelegt werden, wie stark gesperrt
wird. Hier zwei Beispiele:
\begin{center}
  \begin{tabular}{ll}
    \verb!\textls[ 50]{\textsc{schön.}}! & \textls[ 50]{\textsc{schön.}}\\
    \verb!\textls[200]{\textsc{schön?}}! & \textls[200]{\textsc{schön?}}\\
  \end{tabular}
\end{center}
Um hingegen die Hevorhebungen für Überschriften festzulegen (etwa durch
\verb!\setkomafont! bei Verwendung der \acr{KOMA}-Klassen), gibt man an, welche
Schriftschnitte verwendet werden. Dies geschieht mit \verb!\itshape!,
\verb!\bfseries! oder \verb!\scshape!. Um Versalien zu nutzen, kann der Befehl
\verb!\MakeUppercase! eingesetzt werden.

%%% Local Variables:
%%% mode: latex
%%% TeX-master: "main"
%%% End:

\chapter{Schriften}
\label{ch:Schriften}

Das grundlegende Arbeitsmittel der Typographie ist natürlich die
\emph{Schrift}.  Mit ihr wollen wir uns in diesem Kapitel umfassend
beschäftigen.  Einige Aspekte haben wir schon gesehen; hier gehen wir
aber weiter ins Detail.  Da sich die Gestaltung von Schrift »im
Kleinen« abspielt (im Gegensatz zur Gestaltung von bspw. ganzen
Seiten), wird das entsprechende Teilgebiet der Typographie auch als
\emph{Mikrotypographie} bezeichnet.

\section{Begriffliches und Klassifikation}

Eine \emph{Glyphe} ist die konkrete graphische Realisierung eines
(abstrakten) Schriftzeichens.  So sind bspw. \Char{a},
\Char{\textlarger{a}}, \Char{\textsmaller{a}}, \Char{\textit{a}},
\Char{\textbf{a}}, \Char{\textsf{a}}, \Char{\textsf{\textit{a}}} alles
unterschiedliche Glyphen, die den \emph{lateinischen Kleinbuchstaben
  a} darstellen.

Zusammenhängende Texte werden in einer \emph{Schriftart} gesetzt; die
darin vorhandenen Glyphen sind alle ähnlich und zueinander passend
gestaltet~-- sonst ergibt sich {U\fontfamily{ptm}\selectfont
  n\fontfamily{pplj}\selectfont r\fontfamily{lmr}\selectfont
  u\fontfamily{uop}\selectfont h\fontfamily{pcr}\selectfont e}.  Liegt
eine Schriftart in verschiedenen Varianten vor, was üblicherweise der
Fall ist, so werden diese \emph{Schriftschnitte} genannt
(s. \cref{sec:Schnitte}).  Eine solche Schriftart mit mehreren
Schnitten nennt man dann auch eine \emph{Schriftfamilie}.\footnote{Das
  ist das Beste, was ich (Philip) nach längerer Recherche zur
  Unterscheidung der Begriffe \emph{Schriftart} und \emph{-familie}
  herausbekommen zu haben glaube.  Allerdings fällt mir spontan
  (während ich diesen Text schreibe) kein Beispiel für eine Schriftart
  ein, von der es nur einen Schnitt gibt, die also keine Familie ist.}

Schriftarten werden nach Aspekten ihrer Gestaltung klassifiziert.  Wir
beschäftigen uns hier nur mit \emph{Satzschriften} (also keine
Handschriften) für das lateinische Alphabet.  Diese werden
grundsätzlich unterschieden in zwei \emph{Schriftgattungen}, nämlich
\emph{Antiqua-Schriften}\footnote{Auch für das griechische, armenische
  und kyrillische Alphabet sind Antiqua-Schriften heutzutage üblich.}
einerseits und \emph{gebrochene Schriften} andererseits: in
gebrochenen Schriften haben die Bögen der Buchstaben »Knicke«; in
Antiqua-Schriften sind sie rund.  Heutzutage werden fast alle
Druckerzeugnisse im lateinischen Alphabet in Antiqua-Schriften
gesetzt, gebrochene Schriften (bspw. \emph{Textura},
\emph{Schwabacher}, \emph{Fraktur}) sind veraltet.

Historisch sind die Antiqua-Schriften im 15.\,Jahrhundert entstanden,
indem die Großbuchstaben der antiken \emph{Capitalis monumentalis}
--~der Schrift römischer Steininschriften~-- mit den Kleinbuchstaben
der \emph{humanistischen Minuskel} kombiniert wurden.  Für
handgeschriebene Bücher waren vorher reine Minuskelschriften üblich,
die sich ab Ende der Antike ursprünglich als Handschriften entwickelt
hatten; höchstens einzelne Anfangsbuchstaben wurden als ausgeschmückte
\emph{Initialen} in Großbuchstaben geschrieben.\footnote{Die
  englischen Bezeichnungen \foreignlanguage{british}{uppercase} und
  \foreignlanguage{british}{lowercase} für Groß- bzw. Kleinbuchstaben
  kommen aus der Zeit des Bleisatzes, als die Lettern für
  Großbuchstaben in Setzkästen in der oberen und die für
  Kleinbuchstaben in der unteren Schublade aufbewahrt wurden.}

Innerhalb der Gattung der Antiqua-Schriften werden Schriftarten weiter
in kleineren »Taxa« (um mal ein Wort aus der Biologie
zweckzuentfremden) klassifiziert:\footnote{In der Zeit des Bleisatzes
  war in Deutschland die Klassifikation gemäß \acr{DIN}~16518 üblich
  (siehe \url{https://de.wikipedia.org/wiki/DIN_16518}); diese ist
  aber einerseits kunsthistorisch fragwürdig und andererseits
  veraltet.  Wir orientieren uns im Folgenden grob an der Matrix
  Beinert (siehe
  \url{https://www.typolexikon.de/schriftklassifikation-matrix-beinert/}).}
Die Gattung wird unterteilt in \emph{Hauptschriftgruppen} und diese
jeweils in \emph{Schriftuntergruppen}.  Die Hauptschriftgruppen
unterscheiden sich dabei durch die Art der an den Glyphen vorhandenen
\emph{Serifen}, den Linien, die die Hauptstriche »abschließen«.  Die
\todo{Abbildung (vergrößert) mit Serifen?}
Klassifikation ist wie folgt:
\begin{itemize}
\item Eine Schrift mit »normalen«, nicht betonten Serifen heißt
  \emph{Antiqua} bzw. englisch \emph{\foreignlanguage{british}{serif}}
  oder \emph{\foreignlanguage{british}{roman}} (ja, der Name der
  Hauptschriftgruppe ist derselbe wie der der Gattung; diese
  Schriftarten waren nämlich die ursprünglichen Antiqua-Schriften).
  Nach ihrem historischen Entstehungszeitpunkt werden die folgenden
  drei Schriftuntergruppen unterschieden:
  \begin{itemize}
  \item \emph{Renaissance-Antiqua}, mit den Schriftnebengruppen
    \emph{Venezianische} und \emph{Französische Renaissance-Antiqua}:
    bspw. {\fontfamily{EBGaramond-OsF}\selectfont Garamond},
    {\fontfamily{pplj}\selectfont \textsmaller[.5]{Palatino}}, % URW Palladio L 
    Minion
  \item \emph{Vorklassizistische Antiqua}\footnote{Nach
      \acr{DIN}~16518: »Barock-Antiqua« (irreführend)}:
    bspw. {\fontfamily{Baskervaldx-OsF}\selectfont Baskerville},
    {\fontfamily{ptm}\selectfont Times} % Nimbus Roman No. 9 L
  \item \emph{Klassizistische Antiqua}:
    bspw. {\fontfamily{bodoni}\selectfont Bodoni},
    {\fontfamily{TheanoDidot-TOsF}\selectfont Didot},
    {\fontfamily{lmr}\selectfont Computer Modern / Latin Modern}
  \end{itemize}
\item Eine Schrift mit betonten Serifen heißt \emph{Egyptienne} bzw.
  englisch \emph{\foreignlanguage{british}{slab serif}}.\footnote{Nach
    \acr{DIN}~16518: »Serifenbetonte Linear-Antiqua«}
  (Schriftuntergruppen sind uns hier egal.)  Bspw.
  {\fontfamily{bch}\selectfont Bitstream Charter},
  {\fontfamily{ccr}\selectfont Concrete Roman},
  {\fontfamily{ucr}\selectfont Courier} % Nimbus Mono L
\item Eine Schrift ohne Serifen heißt \emph{Grotesk} bzw. englisch (ja
  ja, mit Französisch) \emph{\foreignlanguage{british}{sans
      serif}}.\footnote{Nach \acr{DIN}~16518: »Serifenlose
    Linear-Antiqua«} (Auch hier sind uns Schriftuntergruppen egal.)
  Bspw.  {\fontfamily{phv}\selectfont \textsmaller[.5]{Helvetica}}, % Nimbus Sans L
  {\fontfamily{pag}\selectfont \textsmaller[.5]{Avant Garde}}, % Gothic L
  {\fontfamily{ua1}\selectfont \textsmaller[.5]{Arial}} % URW A030
\item Die letzte Hauptschriftgruppe der Antiqua-Schriften sind
  \emph{Zierschriften}.  Das sind generell alle Antiqua-Schriften, die
  keine Antiqua, Egyptienne oder Grotesk sind. Zu ihnen gehören
  u.\,a. die sog. \emph{Antiqua-Varianten} nach \acr{DIN}~16518.  Ein
  Beispiel für letztere sind sog. \emph{Inzisen}, an Inschriften
  orientierte Schriften mit sehr dezenten Serifen, wie bspw.
  \textsf{\textsmaller[.5]{Optima / Classico}}.\footnote{Die Optima
    wird gerne --~so auch in diesem Skript~-- als Grotesk zu Palatino
    oder Minion verwendet, obwohl sie eigentlich™ keine ist.}
\end{itemize}
Die Gestaltung und Klassifikation von Schriften gleicht einem Fass
ohne Boden; wir gehen hier nicht weiter ins Detail.\footnote{Wer
  möchte, kann dazu sehr viel Weitergehendes auf Wikipedia oder im
  Typo\-lexikon (\url{https://www.typolexikon.de/}) lesen.}

Die Schrift, in der der Großteil eines typographischen Werks gesetzt
wird, wird historisierend die jeweilige \emph{Brotschrift} genannt~--
mit ihr haben Schriftsetzer früher »ihr täglich Brot« verdient.
Angeblich™ lesen sich im Fließtext Schriften mit Serifen leichter als
Grotesken.  Es gibt dazu auch einige Untersuchungen; soweit ich
(Philip) weiß, sind die aber zu keinen aussagekräftigen (sprich
statistisch signifikanten) Ergebnissen gekommen.  (Tatsächlich sind
serifenlose Schriften wohl teilweise hilfreich bei Legasthenie.)  Auf
jeden Fall sind Grotesken eine jüngere Erfindung (sie sind Anfang des
19.\,Jahrhunderts entstanden), und »traditionell« werden gedruckte
Fließtexte in Antiqua gesetzt.  Auf nicht allzu hoch aufgelösten
Pixel-Bildschirmen sind Serifen oft eher schlecht in gleichmäßiger
Qualität darstellbar; daher werden in Computer-Kontexten, bspw. im
Webdesign, häufig auch serifenlose Schriften als Brotschrift
eingesetzt.

\section{Schriftschnitte}
\label{sec:Schnitte}

%% Stärke, Lage, Breite.  Bei Lage (also gerade vs. kursiv) auch
%% Historisches™ erwähnen.

\section{Liniensystem}
\label{sec:Linien}

%% Inkl. Maßeinheiten und so, und Zeilenabstand

%% Zu Schriftgröße: 1em = Schriftgröße.  Was das mit der tatsächlichen
%% Glyphengröße zu tun hat: heutzutage sehr willkürlich; oft so was
%% wie Versalhöhe ≈ .7em (oder so?!).

%% Zeilenabstand: eigentlich sollte™ man einfach den Abstand zwischen
%% den Grundlinien messen.  Es gibt sinnvolle Standardeinstellungen,
%% bspw. in LaTeX 10pt → 12pt, 11pt → 13.6pt, 12pt → 14.5pt.  MS Word
%% und LibreOffice benutzen auch ca. 1,2-faches der Schriftgröße als
%% Grundeinstellung, das heißt da aber »einzeilig«.

\section{Kerning und Spacing}
\label{sec:KernSpace}

%% Hier microtype erwähnen?

\section{Ligaturen}

%% Ligatur = eine Glyphe für mehrere Buchstaben.  Sinnvoll, um
%% ruhigeres Schriftbild zu erreichen.  Traditionelle Ligaturen aus
%% lateinischen Texten; je nach Sprache zusätzliche.  Fun fact: ß =
%% ſ-z- oder ſ-s-Ligatur.  (Problem: die Minion hat kein ſ :( )

%% Ligaturbrecher erwähnen!

\section{Ziffern}
\label{sec:Ziffern}

%% Minuskelziffern, Majuskelziffern, Tabellenziffern / lining figures

\section{Beispiele}

{\fontfamily{pplj}\selectfont Palatino 1234 \emph{kursiv} \textsc{kapitälchen}}
%% Garamond, Palatino, Minion, Optima/Classico, Times, Helvetica

%% Jeweils auch die lustigen Namen nach DIN 16518 erwähnen!

%%% Local Variables:
%%% mode: latex
%%% TeX-master: "main"
%%% End:

\chapter{Seitenlayout}

%% - Satzspiegel/Zeilenlänge
%% - Zeilenabstand
%% - Seitenumbrüche
%% - Schriftgrößen mischen
%% - Kopf-/Fußzeilen/Pagination
%% - andere Formate (z.B. Plakate, Titelseiten, …)
%% - Blocksatz, Silbentrennung

%%% Local Variables:
%%% mode: latex
%%% TeX-master: "main"
%%% End:


\end{document}

%%% Local Variables:
%%% mode: latex
%%% TeX-master: t
%%% End:
